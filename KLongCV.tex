\documentclass[9pt,a4paper]{moderncv}

% moderncv themes
\moderncvtheme[blue]{classic}                  % optional argument are 'blue' (default), 'orange', 'green', 'red', 'purple', 'grey' and 'roman' (for roman fonts, instead of sans serif fonts)
%\moderncvtheme[green]{classic}                % idem

% character encoding
\usepackage[utf8]{inputenc}                   % replace by the encoding you are using

% adjust the page margins
\usepackage[scale=0.89]{geometry}
\setlength{\hintscolumnwidth}{2.5cm}						% if you want to change the width of the column with the dates
\AtBeginDocument{\setlength{\makecvheadnamewidth}{14 cm}}  % only for the classic theme, if you want to change the width of your name placeholder (to leave more space for your address details
\AtBeginDocument{\recomputelengths}                     % required when changes are made to page layout lengths


% personal data
\firstname{Kenneth}
\familyname{Long}
\title{Physicist and data scientist}               % optional, remove the line if not wanted


\address {\textbf{Nationality}:  American}{\textbf{Birth date}: May 8, 1991}
\extrainfo{Route de Champvigny 51, 1242 Satigny (Switzerland)}
\mobile  {+41 (0)77 503 9513}         % optional, remove the line if not wanted
%\phone{            (+39)062291044      }                      % optional, remove the line if not wanted                  % optional, remove the line if not wanted
\email{kdlong07@gmail.com}  
\social[linkedin]{kdlong07}
\social[github]{kdlong}

\photo[64pt][0.4pt]{photo}                         % '64pt' is the height the picture must be resized to, 0.4pt is the thickness of the frame around it (put it to 0pt for no frame) and 'picture' is the name of the picture file; optional, remove the line if not wanted


% to show numerical labels in the bibliography; only useful if you make citations in your resume
\makeatletter
\renewcommand*{\bibliographyitemlabel}{\@biblabel{\arabic{enumiv}}}
\makeatother

% bibliography with mutiple entries
%\usepackage{multibib}
%\newcites{book,misc}{{Books},{Others}}

%\nopagenumbers{}                             % uncomment to suppress automatic page numbering for CVs longer than one page
%----------------------------------------------------------------------------------
%            content
%----------------------------------------------------------------------------------

\begin{document}
\maketitle

%\section{Personal Details}
%\cvitem{First name:}{Claudia}
%\cvitem{Last name:}{Tambasco}
%\cvitem{Nationality:}{Italian}
%\cvitem{Date of birth:}{28th May 1988}
%\cvitem{Country of birth:}{Italy}
%\cvitem{Place of birth:}{Rome (Italy)}
%\cvitem{Home address:}{Via F.Vitalini, 65  E/D}
%\cvitem{}{00155 Roma}

\vspace{-1cm}
\section{Skills}

\cvline{Programming Languages}{Highly experienced in Python, Bash, C++, C;
    comfortable in SQL, R, Java, Scala; familiar with Perl, Fortran}
\cvline{Analysis}{Linear/logistic regression, mixed linear models, deep learning, tree-based methods}
\cvline{Tools}{Machine learning (Keras, Tensorflow, PyTorch), visualisation (matplotlib, plotly, ggplot), data wrangling (pandas, numpy), version control (git), continuous integration (GitHub CI), batch computing (HTCondor), data bases (SQL)}
\cvline{Communication}{Experienced communicating work to diverse audiences. Presented $\sim$20 talks at international conferences. Seminars and volunteer efforts communicating science to general public.}

\section{Experience}
\cventry{Aug 2019 - present}{Postdoctoral physicist and data scientist}{Massachusetts Institute of Technology (MIT) and European Organization for Nuclear Research (CERN)}{Geneva, CH}{}
{
\begin{itemize}
    \item Developed and deployed graph neural network with custom architecture for fast and accurate clustering of thousands of electrical signals into collections of a common origin,
        improving cluster accuracy over algorithmic clustering by up to 25\% (publication 1). Built visualisation framework in numpy, pandas, and plotly to assess performance.
    \item Leader of a 20 person development team maintaining and developing clustering algorithms to convert detector signals to high-level representations of particle collisions.
    \item Leading effort to precisely extract data set properties via complex maximum-likelihood fits with thousands of nuisance parameters. Contributor to a custom TensorFlow-based framework allowing an order of magnitude speedup in parameter extraction with respect to previous C++-based minimizer.
    \item Principle developer, repository maintainer, and reviewer of analysis framework used by $\sim$10 collaborators. Implemented self-hosted continuous integration in GitHub for automated testing and validation of full analysis pipeline, significantly improving development robustness and reducing review overhead.
    \item Designed and developed production framework for a lightweight and versatile columnar data storage format used by hundreds of scientists for faster simulation and analysis. By pruning little-used attributes and optimizing compression, reduced storage footprint by an order of magnitude while improving user friendliness. Recognized with a collaboration-wide award for this work.
    \item Coordinated group of $\sim$50 colleagues making measurements of predicted, but previously unmeasured, phenomena. Lead review of 15-20 projects, resulting in $\sim$20 publications over two years.
\end{itemize}
}

\cventry{Jul 2013 - Jul 2019}{Researcher}{University of Wisconsin-Madison}{Madison, WI, USA and Geneva, CH}{}
{
\begin{itemize}
    \item Analyzed data set of billions of particle collisions. Used statistical modeling and maximum likelihood estimation to establish properties of a rare process and their statistical significance.
    \item Developed, validated, and deployed scalable software frameworks to perform big data analysis using
distributed computing infrastructure (100 GB--1 PB pipelines), used by 10-20 people.
    \item Lead a small group ($\sim$10 people) on the support, software integration, and validation of new simulation tools, exploiting expertise in software, numerical calculations, and distributed computing.
    \item Instructor for two classes of electricity and magnetism for engineers ($\sim$50 students). Prepared 4 lectures/week and 2 labs/week, wrote and graded assignments.
\end{itemize}
}
\section{Education}
%\cventry{year--year}{Degree}{Institution}{City}{\textit{Grade}}{Description}
\cventry{July 2013 -- Apr 2019}{Ph.D. in Physics}{University of Wisconsin-Madison}{Madison, WI, USA}{}
{
    Specialization: Experimental particle physics.
    \normalsize \textit{GPA: 3.73/4.00} 
}
\cventry{Aug 2009 -- May 2013}{Bachelor of Science in Physics}{Tennessee Technological University}{Cookeville, TN, USA}{}{\normalsize \textit{GPA: 3.97/4.00}}

\pagebreak
\section{Publications}
\href{https://inspirehep.net/authors/1280606}{Hundreds of peer-reviewed publications} in major journals as part of the CMS Collaboration at CERN ($\sim$10 as principle editor), as well as several limited-author publications on use of graph neural networks for clustering tasks in particle physics, including:
\begin{itemize}
    \item S. Qasim, N. Chernyavskaya, K. Long, J. Kieseler, O. Viazlo, M. Pierini, N. Raheel, ``End-to-end multi-particle reconstruction in high occupancy imaging calorimeters with graph neural networks,'' Eur. Phys. J. C 82, 753 (2022), \href{https://arxiv.org/abs/2106.01832}{\texttt{arXiv:2106.01832 [ins-det]}}.
\end{itemize}

\section{Languages}
\cvlanguage{English}{Native}{}
\cvlanguage{French}{Professionally proficient}{Completed CEFR B2 certification course winter 2021}

\section{Personal skills}
\cvline{}{Extensive project management experience, working with and leading diverse groups of collaborators. Experienced at summarizing complex topics to expert and non-expert audiences.}
\end{document}
