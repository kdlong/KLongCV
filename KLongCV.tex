\documentclass[10pt]{res} % default is 10 pt
%\usepackage{helvetica} % uses helvetica postscript font (download helvetica.sty)
%\usepackage{newcent}   % uses new century schoolbook postscript font 
%\setlength{\textheight}{9 in} % increase text height to fit resume on 1 page
%\newsectionwidth{0pt}  % So the text is not indented under section headings
\usepackage[dvipsnames]{xcolor}
\usepackage{hyperref}
\usepackage[T1]{fontenc}
\usepackage{enumitem}
\setlist[itemize]{align=parleft,left=0pt..1em, label={$-$}}
\setlist[enumerate]{align=parleft,left=0pt..1em}

\begin{document}

\name{Kenneth Long\\ \hspace{-0.2cm}{kenneth.long@cern.ch} \\[11pt]} % the \\[12pt] adds a blank line after name

\address{{\bf Current Address} \\  CERN B40/2-A20 \\  Geneva 23 \\ Switzerland \\ Tel: +41 77 501 95 13 }
\address{}

\begin{resume}
\section{\textsc{Education}}
  \textbf{University of Wisconsin-Madison}, Madison, WI \\
Ph.D. in Experimental High Energy Physics, April 2019 \\
  Thesis advisor: Prof. Matthew Herndon \\
    Thesis topic: \emph{Measurement of electroweak WZ boson production and search for new physics in proton-proton collisions at $\sqrt{s}=13$\,TeV with the CMS detector at the CERN LHC} \\
  \textbf{Tennessee Technological University}, Cookeville, TN \\
Bachelor of Science in Physics, \textit{Summa Cum Laude}, May 2013

\section{\textsc{Professional Experience}}
\vspace{-0.1in}
\begin{tabbing}
\hspace{2.3in}\= \hspace{2.4in}\= \kill % set up two tab positions
{\bf Senior Research Fellow} \>\> Aug. 2019 -- Present \\
European Organization for Nuclear Research (CERN)   \\  Meyrin, Switzerland \\
{\bf Doctoral Researcher} \>\> Jul. 2013 -- Jul. 2019\\
Univ. of Wisconsin-Madison Compact Muon Solenoid Group \\  Madison, WI, US and Meyrin, Switzerland \\
\end{tabbing}\vspace{-20pt}      % suppress blank line after tabbing

\section{\textsc{Research Interests}}
Testing the standard model of particle physics in collisions at the Large Hadron Collider (LHC). 
Leveraging precision measurements as a tool to probe subtle effects from undiscovered new phenomena. 
Developing new machine-learning based reconstruction techniques to enable future precision measurements in the 
complex environment of future colliders.

\section{\textsc{Leadership Roles in the CMS experiment}}
\begin{itemize}
\item{\textbf{Convener of Particle Flow Reconstruction Group} \hfill{June 2019 -- Present}} \\
    Coordinating group of around 30 members maintaining and improving global event reconstruction algorithms
    used throughout CMS. Focused on maintaining Run 2 performance for the LHC Run 3 while exploring
    novel improvements, especially via machine learning, to enable fast clustering and pattern identification
    for the HL-LHC.
\item{\textbf{Convener of Multi-boson Analysis Group} \hfill{Sep. 2019 -- Present}} \\
    Coordinating standard model physics subgroup studying production of final states
    with multiple vector bosons. Oversaw the development and publication of over ten new measurements, 
    including the first observation of triple vector boson production and the most accurate measurements 
    to date of WZ, W$^{\pm}$W$^{\pm}$, and W$\gamma$ vector boson scattering. Currently reviewing and guiding
    more than ten on-going analyses.
\item{\textbf{Convener of Matrix Element and Future Generators Group} \hfill{Jun. 2017 -- May 2019}}\\
    Directed Monte Carlo Generators subgroup 
    responsible for maintenance and development of simulation software used to simulate
    billions of collisions for use in CMS analyses.
\item{\textbf{Standard Model Physics Group Monte Carlo Contact} \hfill{Feb. 2016 -- Aug. 2017}} \\
    Oversaw
    production of Monte Carlo event samples for the CMS standard model physics group. 
    Duties included technical support, coordinating requests, and monitoring samples in computing production.
\item{\textbf{ Data Certification Expert, Cathode Strip Chamber}\hfill{Jul. 2015 -- Nov. 2018}} \\
    Coordinated discussion and determined if
    detector data quality was satisfactory for use in physics analyses.
    Communicated with experts to understand
    causes of data quality degradation.
\end{itemize}

\section{\textsc{CMS Analysis Activities}}
\begin{itemize}
  \item\textbf{{Precision measurements of W and Z boson production}} \\
    Performing measurements of W and Z boson properties in high- and low-pileup data.
    Integrated software, performed extensive validation, and produced nearly 2 billion simulated
    W and Z events with the state-of-the-art MiNNLO program at NNLO in QCD,
    significantly reducing Monte Carlo statistical and modeling uncertainties
    for the W mass, weak mixing angle, and other precision measurements.
    Developed technical implementation and analysis strategy to match MiNNLO predictions to
    highest-accuracy resummation results at N$^{3}$LL, 
    reducing the modeling uncertainty by a further 10-20\% in the region most relevant for the W mass measurement.
    Leading contributor to a new analysis framework for the W mass measurement capable of analyzing billions of events in minutes.
    
  \item\textbf{{Measurements of diboson production via vector boson scattering}} \\
    Analysis contact and lead analyzer for first WZ vector boson scattering analysis in CMS (CMS publication 4).
    Responsible for all aspects, 
    including development of analysis framework, procedure, optimization, and validation. 
    Closely involved in the development of the extended study (CMS publication 3), including 
    development of a novel approach
    to simultaneously measure the WZ and W$^{\pm}$W$^{\pm}$ electroweak cross sections.
    Projections of CMS electroweak WZ production measurement to high luminosity, including studies of
    WZ polarization and sensitivity to new physics (CMS publication 5, other publication 3). 

  \item\textbf{{Measurements of diboson production cross sections}} \\
    First cross section measurements
    of ZZ and WZ production at 13 TeV (CMS publications 7 and 8). 
    Differential measurements and constraints on anomalous
    triple gauge couplings in ZZ channel using leptonic decays
    with 2016 (CMS publication 6) and 2016--2018 datasets (CMS publication 2). 
    Developed analysis software framework, produced results for theoretical comparisons, 
    and supervised graduate students.
    
\end{itemize}

\section{\textsc{Other Projects and Responsibilities in CMS}}
\begin{itemize}
  \item\textbf{Clustering and reconstruction algorithms with machine learning}\hfill{Oct. 2019 -- present} \\
    Developing graphical neutral networks for end-to-end particle flow reconstruction,
    particularly focused on applications in the high-granularity calorimeter planned for the CMS Run 4 upgrade.
    First use of the object condensation approach to one-step clustering and regression,
    integrated with CMS software and simulation (publication 1).
    Improved use of simulation software to define robust target truth for particle-flow reconstruction.
    Developed flexible framework for 3D visualization of existing and idealized reconstruction algorithms.

  \item\textbf{Monte Carlo generator integration and support}\hfill{Jan. 2017 -- present} \\
    Developed and implemented the production of a new data format, now used throughout CMS, to enable fast and lightweight production
    of generator-level collision simulations. Performed major restructuring of CMS simulation software interface
    to systematic variations, improving ease of use and enabling more extensive evaluation of theoretical uncertainties in analysis.
    Leading developer of software used to interface simulation tools
    developed by theoretical collaborations into experimental software framework (2017 -- 2019).

  \item\textbf{Endcap muon detector detector on-call shifter}\hfill{Apr. 2016 -- Jul. 2018} \\
    First contact for CMS endcap muon system during commissioning and data-taking for three weeks per year
    from 2016-2018. Duties included monitoring of the cathode strip chamber system, coordinating maintenance activities,
    and addressing any urgent issues affecting data collection or quality.

  \item\textbf{Muon detector longevity studies}\hfill{Jan. 2016 -- Jun. 2016} \\
    Assisted in setup of test system and performance measurements of cathode strip chambers
    exposed to high-intensity gamma radiation to assess the physical degradation and impact on muon
    reconstruction performance. Operation shifts during data collection with muon beams.

\end{itemize}

\section{\textsc{CMS Publications and Public Results with Significant Contributions}}
\begin{enumerate}
  \item S. Qasim, K. Long, J. Kieseler, and M. Pierini for the CMS Collaboration, R. Nawaz, ``Multi-particle reconstruction in the High Granularity Calorimeter using object condensation and graph neural networks,'' \href{https://arxiv.org/abs/2106.01832}{\texttt{arXiv:2106.01832 [ins-det]}} 
  \item CMS Collaboration, ``Measurements of pp $\rightarrow$ ZZ production cross section and constraints on anomalous triple gauge couplings at $\sqrt{s} = 13~\mathrm{TeV}$,'' Eur. Phys. J. C 81 (2021) 200, \href{https://arxiv.org/abs/2009.01186}{\texttt{arXiv:2009.01186 [hep-ex]}}
  \item CMS Collaboration, ``Measurements of production cross sections of WZ and same-sign WW boson pairs in association with two jets in proton-proton collisions at $\sqrt{s} =$ 13 TeV'' Phys. Lett. B 809 (2020) 135710, \href{https://arxiv.org/abs/2005.01173}{\texttt{arXiv:2005.01173 [hep-ex]}}
  \item CMS Collaboration, ``Measurement of electroweak WZ boson production and search for new physics in WZ $+$ two jets events in pp collisions at $\sqrt{s}=13$\,TeV,'' Phys. Lett. B 795 (2019) 281, \href{https://arxiv.org/abs/1901.04060} {\texttt{arXiv:1901.04060 [hep-ex]}}
  \item CMS Collaboration, ``Prospects for the measurement of electroweak and polarized $\mathrm{WZ}\to3\ell\nu$ production cross sections at the High-Luminosity LHC,'' CMS-PAS-FTR-18-038
  \item CMS Collaboration, ``Measurement of the pp $\rightarrow$ ZZ production cross section, $\mathrm{Z} \to 4\ell$ branching fraction, and constraints on anomalous triple gauge couplings at $\sqrt{s} = 13~\mathrm{TeV}$,'' Eur. Phys. J. C 78 (2018) 165, \href{https://arxiv.org/abs/1709.08601}{\texttt{arXiv:1709.08601 [hep-ex]}}
  \item CMS Collaboration, ``Measurement of the WZ production cross section in pp collisions at $\sqrt{s}$ = 13 TeV,''
Phys. Lett. B 766, 268 (2016), \href{https://arxiv.org/abs/1607.06943}{\texttt{arXiv:1607.06943 [hep-ex]}}
  \item CMS Collaboration, ``Measurement of the ZZ production cross section and Z $\rightarrow l^{+}l^{-}l'^{+}l'^{-}$ branching fraction in pp collisions at $\sqrt{s} =$ 13 TeV,''
Phys. Lett. B 763, (2016) 280, \\ \href{https://arxiv.org/abs/1607.08834} {\texttt{arXiv:1607.08834 [hep-ex]}}
  \item CMS Collaboration, ``Performance of the CMS muon detector and muon reconstruction with proton-proton collisions at $\sqrt{s} = 13$ TeV,''
JINST 13 (2018) P06015, \href{https://arxiv.org/abs/1804.04528} {\texttt{arXiv:1804.04528 [hep-ex]}}
\end{enumerate}

\section{\textsc{Other Publications with Significant Contributions}}
\begin{enumerate}
  \item D. Franzosi et al.,``Vector Boson Scattering Processes: Status and Prospects,''
    Submitted to Reviews in Physics, VBSCAN-PUB-04-21, \href{https://arxiv.org/abs/2106.01393}{\texttt{arXiv:2106.01393 [hep-ph]}}
  \item M. Gallinaro, K. Long, J. Reuter, R. Ruiz (editors) et al.,``Beyond the Standard Model in Vector Boson Scattering Signatures,''
    VBSCAN-PUB-04-20, CERN-OPEN-2020-008, \href{https://arxiv.org/abs/2005.09889}{\texttt{arXiv:2005.09889 [hep-ph]}}
  \item S. Amoroso et al., ``Les Houches 2019: Physics at TeV Colliders Standard Model Working Group Report,''
    \href{https://arxiv.org/abs/2003.01700 } {\texttt{arXiv:2003.01700 [hep-ph]}}
  \item HL-LHC Collaboration and HE-LHC Working Group, ``Standard Model Physics at the HL-LHC and HE-LHC,''
    CERN-LPCC-2018-03, \href{https://arxiv.org/abs/1902.04070}{\texttt{arXiv:1902.04070 [hep-ph]}}
  \item J.R. Andersen et al., ``Les Houches 2017: Physics at TeV Colliders Standard Model Working Group Report,''
    FERMILAB-CONF-18-122-CD-T, UWTHPH-2018-5, \href{https://arxiv.org/abs/1803.07977}{\texttt{arXiv:1803.07977 [hep-ph]}}
  \item C. Anders et al., ``VBSCan Split 2017 Workshop Summary,'' 
    VBSCAN-PUB-01-17, FERMILAB-CONF-18-021-PPD, \href{https://arxiv.org/abs/1801.04203}{\texttt{arXiv:1803.07977  [hep-ph]}}
\end{enumerate}

\section{\textsc{Presentations at International Conferences and Workshops}}
\vspace{-0.1in}
\begin{tabbing}
\hspace{2.3in}\= \hspace{2.6in}\= \kill % set up two tab positions
\bf{Vector boson scattering for SNOWMASS---VBSCan workshop} 		 \> \>	    Feb. 2021 \\
Virutal -- Invited Plenary\\
``Overview of current CMS results on Vector boson scattering'' \\
\bf{BSM models in vector boson scattering---VBSCan workshop} 		 \> \>	    Dec. 2019 \\
Lisbon, Portugal -- Invited Plenary\\
``Vector boson scattering results and future prospects'' \\
\bf{Standard Model at LHC 2019} 		 \> \>	    Apr. 2019 \\
Zurich, Switzerland -- Contributed Plenary \\
``Recent vector boson fusion and scattering measurements at ATLAS and CMS'' \\
\bf{US LHC Users Association Meeting 2018} 		 \> \>	    Oct. 2018 \\
Fermilab, Batavia, IL, US -- Young Scientists Lightning Round \\
``Search for electroweak WZ vector boson scattering and new physics at CMS'' \\
\emph{Awarded best presentation } \\
\bf{Multi-boson Interactions 2018} 		 \> \>	    Aug. 2018 \\
Ann Arbor, MI, US -- Contributed Plenary \\
``Results from ATLAS and CMS on the neutral VBS production \\
\bf{XXXIX International Conference on High Energy Physics} 		 \> \>	    Jul. 2018 \\
Seoul, South Korea -- Contributed Parallel\\
``Vector Boson Scattering Results from CMS'' \\
\bf{30th Rencontres de Blois} 		 \> \>	    Jun. 2018 \\
Blois, France -- Contributed Parallel\\
``VBS and VBF results from ATLAS and CMS'' \\
\bf{17th MCnet Collaboration Meeting} 		 \> \>	    Apr. 2018 \\
CERN, Geneva, Switzerland -- Contributed Plenary\\
``Use of Monte Carlo generators in CMS'' \\
\bf{Large Hadron Collider Physics 2017} 		 \> \>	    May 2017 \\
Shanghai, China -- Contributed Parallel \\
``Multiboson Results from CMS'' \\
\bf{Multi-boson Interactions 2016} 		 \> \>	    Aug. 2016 \\
Madison, WI, US -- Contributed Plenary \\
``Resent results from ATLAS and CMS on the VV+jets production'' \\
\bf{Division of Nuclear Physics, American Physical Society Fall Meeting} 		 \> \>	    Fall 2012 \\
Newport Beach, CA, US -- Poster Presentation\\
``Materials testing and performance optimization for the SAMURAI-TPC'' \\
\bf{Division of Nuclear Physics, American Physical Society Fall Meeting} 		 \> \>	    Fall 2011 \\
East Lansing, MI, USA -- Poster Presentation\\ 
``Creation of Thin Deuterated Polyethylene Targets for Inverse Kinematics \\Transfer Reaction Measurements'' \\
\end{tabbing}\vspace{-20pt}      % suppress blank line after tabbing

\section{\textsc{Invited Seminars}}
\vspace{-0.1in}
\begin{tabbing}
\hspace{2.3in}\= \hspace{2.6in}\= \kill % set up two tab positions
\textbf{Brookhaven National Laboratory Particle Physics Seminar} \>\> Jan. 2019\\ 
``Measurements and new physics searches with WZ vector boson scattering events at CMS'' \\
Brookhaven National Laboratory \\
Brookhaven, NY, US \\
\textbf{Tennessee Tech University Research Seminar for Undergraduates} \>\> Nov. 2018\\ 
``Searching for new physics (and making measurements) at the CERN Large Hadron Collider'' \\
Tennessee Technological University
Cookeville, TN, US \\
\end{tabbing}\vspace{-20pt}      % suppress blank line after tabbing

%\section{Other Experience}

\section{\textsc{Other Professional Experience}}
\vspace{-0.1in}
\begin{tabbing}
\hspace{2.3in}\= \hspace{2.5in}\= \kill % set up two tab positions
  \textbf{Research Experience for Undergraduates Student Researcher} \>\>{May 2012--Aug. 2012}
\end{tabbing}\vspace{-20pt}      % suppress blank line after tabbing
National Superconducting Cyclotron Laboratory, Symmetry Energy Project Group \\
Michigan State University, East Lansing, MI \\
Assisted in construction and design of the SAMURAI Time Projection Chamber. 
Particularly involved in selecting conductive materials for use in active 
volume of detector. Researched and tested products for performance and lack 
of contaminants. Also focused on optimizing performance of the TPC gating grid circuit.

\vspace{-0.1in}
\begin{tabbing}
\hspace{2.3in}\= \hspace{2.5in}\= \kill % set up two tab positions
\textbf{Undergraduate Research Assistant, Astrophysics Group} \>\> May 2011--Aug. 2011
\end{tabbing}\vspace{-20pt}      % suppress blank line after tabbing
Holifield Radioactive Ion Beam Facility \\
Oak Ridge, TN \\
Preparation of deuterated polyethylene targets for inverse kinematics 
transfer-reaction measurements. Research into improvement of 
target making process with emphasis on creation of thinner targets. 
Also assisted in preparation of experiment and covering of shifts 
during experiment.

\section{\textsc{Awards}}
\vspace{-0.1in}
\begin{tabbing}
\hspace{2.3in}\= \hspace{2.6in}\= \kill % set up two tab positions
\textbf{CMS Collaboration 2020 Achievement Award} 				\>\>{Winter 2020}\\
For outstanding contributions to the Monte Carlo simulation group \\
\textbf{Tennessee Technological University Physics Award} 				\>\>{Spring 2013 }\\
Awarded to top graduating physics student \\
\textbf{Undergraduate Physics Honor Society} \>\>{Spring 2012 }\\
Sigma Pi Sigma Tennessee Technological University Chapter 	    \\
\end{tabbing}\vspace{-20pt}      % suppress blank line after tabbing

\section{\textsc{Teaching}}
\vspace{-0.1in}

\begin{tabbing}
\hspace{3.5 in}\= \hspace{1.4in}\= \kill % set up two tab positions
\textbf{Teaching Assistant}, University of Wisconsin-Madison, WI, US \>\> Fall 2013 \\
General Physics 202: Electricity and Magnetism for Engineers \\
TA duties included leading weekly discussions and labs for 50 students. \\
\textbf{Teaching Assistant}, Tennessee Technological University, TN, US \>\> 2012--2013 \\
Calculus Based Physics Lab 1 and 2 \\
Instructed and graded physics lab class for 3 semesters (approx. 20 students each) \\
\end{tabbing}\vspace{-20pt}      % suppress blank line after tabbing

\section{\textsc{Skills}}
\begin{itemize}
\item \textbf{Languages:} English (Native), French (CEFR B2 level)
\item \textbf{Computer Languages:} Experienced in C++, Python, Bash; comfortable in Java, R, Scala; \\ familiar with Perl, Fortran
\item \textbf{Technical Skills:} Scientific knowledge, statistical analysis and machine learning, software design and maintenance, distributed computing, big data, database use, collaboration and strong interpersonal communication skills
\end{itemize} 

\end{resume}

\end{document}
