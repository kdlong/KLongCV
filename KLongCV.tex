\documentclass[10pt]{res} % default is 10 pt
%\usepackage{helvetica} % uses helvetica postscript font (download helvetica.sty)
%\usepackage{newcent}   % uses new century schoolbook postscript font 
%\setlength{\textheight}{9 in} % increase text height to fit resume on 1 page
%\newsectionwidth{0pt}  % So the text is not indented under section headings
\usepackage[dvipsnames]{xcolor}
\usepackage{hyperref}
\usepackage[T1]{fontenc}

\begin{document}

\name{Kenneth Long\\ \hspace{-0.2cm}{kenneth.long@cern.ch} \\[11pt]} % the \\[12pt] adds a blank line after name

\address{{\bf Current Address} \\  CERN B32/4-B03 \\  Geneva 23 \\ Switzerland}
\address{{\bf Permanent Address} \\ 5710 Lake Mendota Dr. \\ Madison, WI, 53705, US
\\ Tel: +41 77 501 95 13}

\begin{resume}
\section{\textsc{Education}}
  \textbf{University of Wisconsin-Madison}, Madison, WI \\
Ph.D. in Experimental High Energy Physics, April 2019 \\
  Thesis advisor: Prof. Matthew Herndon \\
    Thesis topic: \emph{Measurement of electroweak WZ boson production and search for new physics in proton-proton collisions at $\sqrt{s}=13$\,TeV with the CMS detector at the CERN LHC} \\
\\
  \textbf{Tennessee Technological University}, Cookeville, TN \\
Bachelor of Science in Physics, \textit{Summa Cum Laude}, May 2013

\section{\textsc{Professional Experience}}
\vspace{-0.1in}
\begin{tabbing}
\hspace{2.3in}\= \hspace{2.4in}\= \kill % set up two tab positions
{\bf Senior Research Fellow} \>\> Aug. 2019 -- Present \\
European Organization for Nuclear Research (CERN)   \\  Meyrin, Switzerland \\
{\bf Postdoctoral Research Associate} \>\> May 2019 -- July 2019\\
Univ. of Wisconsin-Madison Compact Muon Solenoid Group \\  Madison, WI, US \\
{\bf Graduate Research Assistant} \>\> Jul. 2013 -- Apr. 2019\\
Univ. of Wisconsin-Madison Compact Muon Solenoid Group at CERN   \\  Madison, WI, US and Meyrin, Switzerland \\
\end{tabbing}\vspace{-20pt}      % suppress blank line after tabbing

\section{\textsc{Research Interests}}
Measurements of particle production in proton--proton (pp) collisions with the CMS detector that are sensitive 
to subtle effects from undiscovered new phenomena and expanding the domain of what can be measured with precision
\begin{itemize}
    \item Classic precision measurements: W and Z boson transverse momentum spectra and the W boson mass 
    \item Frontiers of precision: differential measurements of diboson production 
        and vector boson scattering (VBS) 
    \item Improving performance and software integration of theoretical tools within CMS.
        Enable state-of-the art predictions to be leveraged in experimental analysis and simplify comparisons with predictions
    \item Use of graph neural networks for particle reconstruction at the High-Luminosity LHC with 
    the CMS high-granularity calorimeter, crucial for VBS measurements with forward jets 
\end{itemize}

\section{\textsc{Leadership Roles}}
\begin{itemize}
    \item{\textbf{Convener of Multi-boson Analysis Group} \hfill{Aug. 2019 -- Present}} \\
        Directing CMS standard model physics subgroup (Level 3) studying production of final states
    with multiple vector bosons. Overseeing at least ten analyses exploiting the largest-ever
    pp data set, collected in the LHC Run II from 2016--2018. Reviewing and directing
    analyses with unprecedented precision for diboson production, 
    as well as first observations of triple vector boson production and several VBS production modes.
\item{\textbf{Convener of Matrix Element and Future Generators Group} \hfill{June 2017 -- May 2019}}\\
    Directed CMS Monte Carlo Generators subgroup (Level 3)
    responsible for maintenance and development of external matrix element software, 
    including MadGraph5\_aMC@NLO, POWHEG, Sherpa, and Herwig. Responsible for development and support of sample generation for
    major 2017 Monte Carlo production.
\item{\textbf{Standard Model Physics Group Monte Carlo Contact} \hfill{Feb. 2016 -- Aug. 2017}} \\
    Coordinated
    production of Monte Carlo samples for the CMS Standard Model Physics Group. 
    Duties included resolving technical issues and coordinating sample requests and submission.
\item{\textbf{Cathode Strip Chambers Data Certification Expert}\hfill{July 2015 -- Nov. 2018}} \\
    Coordinated discussion and determined if
    CSC data quality was sufficient for CMS physics analysis.
    Communicated with CSC online and offline experts to understand
    causes of CSC data quality degradation.
\end{itemize}

\section{\textsc{Analysis Activities}}
\begin{itemize}
  \item\textbf{{Precision measurements of W and Z boson production}} \\
    Leading efforts to in differential
    measurement of W and Z transverse momentum spectra using low pile-up data
    collected in 2017. Studying ancillary measurements
    to reduce the impact of theoretical uncertainties in the W mass measurement at CMS.
    
  \item\textbf{{Differential measurements of ZZ$\to4\ell$ production}} \\
    Development of analysis framework and supervision of students.
    Working with theorists to compare measurements with state-of-the-art NNLO QCD+NLO EW
    calculations. Preliminary result (publication 2), full publication expected spring 2020.
  \item\textbf{{First CMS search for electroweak WZ production via vector boson scattering}} \\
    Analysis contact and lead analyzer for first WZ vector boson scattering analysis in CMS (publication 1).
    Cross section measurement, with constrains on dimension-8 
    effective field theory operators and charged Higgs bosons also presented.
    Responsible for all aspects, 
    including development of analysis framework, procedure, optimization, and validation. 

  \item\textbf{{Projections of electroweak WZ production to the HL-LHC}} \\
    Projections of CMS electroweak WZ production measurement to high luminosity, including studies of
    WZ polarization and sensitivity to new physics (publication 3). 
    Modifications and extensions of analysis tools developed for CMS result with 2016 LHC data to 
    HL-LHC simulations. Supervision of junior graduate students.

  \item\textbf{{First 13 TeV measurements of WZ and ZZ cross sections}} \\
    Total cross section measurement 
    of ZZ and WZ production at 13 TeV (Publications 5 and 6) using leptonic decays
    and differential measurements and anomalous
    triple gauge couplings limits in ZZ channel
    with full 2016 dataset (Publication 4). 
    Responsible for acceptance
    and efficiency calculations and validation of all theoretical inputs for both analyses. 
    Developed independent framework for WZ analysis used for cross-checks and validation.

\end{itemize}

\section{\textsc{Other Projects and Responsibilities}}
\begin{itemize}
  \item\textbf{CSC Detector On-Call Shifts}\hfill{Apr. 2016 -- Jul. 2018} \\
    First contact for CSC system during CMS commissioning and data-taking for three weeks per year
    from 2016-2018. Duties included monitoring of the CSC system, coordinating activities of CSC with all of CMS, 
    and addressing any urgent issues affecting CMS data collection or quality.

  \item\textbf{CMS MadGraph5\_aMC@NLO integration and support}\hfill{Jan. 2017 -- Present} \\
    Developer of CMS central scripts
    for configuring and steering MadGraph5\_aMC@NLO, the most-used Monte Carlo generator in CMS. Responsible for 
    providing user support, implementing new developments, 
    and improving usability and computing robustness. Principle developer during production
    of new Monte Carlo samples for 2017 data taking.

  \item\textbf{Multi-boson contact for CMS Cross Section Task Force}\hfill{Oct. 2015 -- Mar. 2017} \\
    Liaison with theory community seeking to understand best-available calculations and 
    their application to CMS Monte Carlo samples. 

  \item\textbf{CSC longevity studies at GIF++}\hfill{Jan. 2016 -- Jun. 2016} \\
    Assisted in setup of test system and performance measurements
    of CSC exposed to high intensity gamma radiation to assess the physical degradation and impact on performance
    of muon reconstruction and triggering at CMS. Primarily responsible for measurements over time of leakage current
    versus instantaneous and accumulated radiation. Contributed to shifts during data collection with muon beams.
\end{itemize}

\section{\textsc{CMS Publications and Public Results with Significant Contributions}}
\begin{enumerate}
  \item CMS Collaboration, ``Measurement of electroweak WZ boson production and search for new physics in WZ $+$ two jets events in pp collisions at $\sqrt{s}=13$\,TeV,'' Phys. Lett. B 795 (2019) 281, \href{https://arxiv.org/abs/1901.04060} {\texttt{arXiv:1901.04060 [hep-ex]}}
  \item CMS Collaboration, ``Measurement of the pp$\to$ZZ production cross section at $\sqrt{s}=13$ TeV with the Run 2 data set,'' \href{http://cms-results.web.cern.ch/cms-results/public-results/preliminary-results/SMP-19-001/index.html} {\texttt{CMS-SMP-19-001 [hep-ex]}}
  \item CMS Collaboration, ``Prospects for the measurement of electroweak and polarized $\mathrm{WZ}\to3\ell\nu$ production cross sections at the High-Luminosity LHC,'' CMS-PAS-FTR-18-038
  \item CMS Collaboration, ``Measurement of the pp $\rightarrow$ ZZ production cross section, $\mathrm{Z} \to 4\ell$ branching fraction, and constraints on anomalous triple gauge couplings at $\sqrt{s} = 13~\mathrm{TeV}$,'' Eur. Phys. J. C 78 (2018) 165, \href{https://arxiv.org/abs/1709.08601}{\texttt{arXiv:1709.08601 [hep-ex]}}
  \item CMS Collaboration, ``Measurement of the WZ production cross section in pp collisions at $\sqrt{s}$ = 13 TeV,''
Phys. Lett. B 766, 268 (2016), \href{https://arxiv.org/abs/1607.06943}{\texttt{arXiv:1607.06943 [hep-ex]}}
  \item CMS Collaboration, ``Measurement of the ZZ production cross section and Z $\rightarrow l^{+}l^{-}l'^{+}l'^{-}$ branching fraction in pp collisions at $\sqrt{s} =$ 13 TeV,''
Phys. Lett. B 763, (2016) 280, \\ \href{https://arxiv.org/abs/1607.08834} {\texttt{arXiv:1607.08834 [hep-ex]}}
  \item CMS Collaboration, ``Performance of the CMS muon detector and muon reconstruction with proton-proton collisions at $\sqrt{s} = 13$ TeV,''
JINST 13 (2018) P06015, \href{https://arxiv.org/abs/1804.04528} {\texttt{arXiv:1804.04528 [hep-ex]}}
\end{enumerate}

\section{\textsc{Other Publications with Significant Contributions}}
\begin{enumerate}
  \setcounter{enumi}{7}
  \item HL-LHC Collaboration and HE-LHC Working Group, ``Standard Model Physics at the HL-LHC and HE-LHC,''
    CERN-LPCC-2018-03, \href{https://arxiv.org/abs/1902.04070} {\texttt{arXiv:1902.04070 [hep-ex]}}
  \item J.R. Andersen et al., ``Les Houches 2017: Physics at TeV Colliders Standard Model Working Group Report,''
    FERMILAB-CONF-18-122-CD-T, UWTHPH-2018-5
  \item C. Anders et al., ``VBSCan Split 2017 Workshop Summary,'' 
    VBSCAN-PUB-01-17, FERMILAB-CONF-18-021-PPD
\end{enumerate}

\section{\textsc{Presentations at International Conferences and Workshops}}
\vspace{-0.1in}
\begin{tabbing}
\hspace{2.3in}\= \hspace{2.6in}\= \kill % set up two tab positions
\bf{BSM models in vector boson scattering---VBSCan workshop} 		 \> \>	    Dec. 2019 \\
Lisbon, Portugal -- Invited Plenary\\
``Current VBS measurements and future prospects at ATLAS and CMS'' \\
\bf{Standard Model at LHC 2019} 		 \> \>	    Apr. 2019 \\
Zurich, Switzerland -- Contributed Plenary \\
``Recent vector boson fusion and scattering measurements at ATLAS and CMS'' \\
\bf{US LHC Users Association Meeting 2018} 		 \> \>	    Oct. 2018 \\
Fermilab, Batavia, IL, US -- Young Scientists Lightning Round \\
``Search for electroweak WZ vector boson scattering and new physics at CMS'' \\
\emph{Awarded best presentation } \\
\bf{Multi-boson Interactions 2018} 		 \> \>	    Aug. 2018 \\
Ann Arbor, MI, US -- Contributed Plenary \\
``Results from ATLAS and CMS on the neutral VBS production \\
\bf{XXXIX International Conference on High Energy Physics} 		 \> \>	    Jul. 2018 \\
Seoul, South Korea -- Contributed Parallel\\
``Vector Boson Scattering Results from CMS'' \\
\bf{30th Rencontres de Blois} 		 \> \>	    Jun. 2018 \\
Blois, France -- Contributed Parallel\\
``VBS and VBF results from ATLAS and CMS'' \\
\bf{17th MCnet Collaboration Meeting} 		 \> \>	    Apr. 2018 \\
CERN, Geneva, Switzerland -- Contributed Plenary\\
``Use of Monte Carlo generators in CMS'' \\
\bf{Large Hadron Collider Physics 2017} 		 \> \>	    May 2017 \\
Shanghai, China -- Contributed Parallel \\
``Multiboson Results from CMS'' \\
\bf{Multi-boson Interactions 2016} 		 \> \>	    Aug. 2016 \\
Madison, WI, US -- Contributed Plenary \\
``Resent results from ATLAS and CMS on the VV+jets production'' \\
\bf{Division of Nuclear Physics, American Physical Society Fall Meeting} 		 \> \>	    Fall 2012 \\
Newport Beach, CA, US -- Poster Presentation\\
``Materials testing and performance optimization for the SAMURAI-TPC'' \\
\bf{Division of Nuclear Physics, American Physical Society Fall Meeting} 		 \> \>	    Fall 2011 \\
East Lansing, MI, USA -- Poster Presentation\\ 
``Creation of Thin Deuterated Polyethylene Targets for Inverse Kinematics \\Transfer Reaction Measurements'' \\
\end{tabbing}\vspace{-20pt}      % suppress blank line after tabbing

\section{\textsc{Invited Seminars}}
\vspace{-0.1in}
\begin{tabbing}
\hspace{2.3in}\= \hspace{2.6in}\= \kill % set up two tab positions
\textbf{Brookhaven National Laboratory Particle Physics Seminar} \>\> Jan. 2019\\ 
``Cross section measurements and new physics searches with WZ vector boson scattering events at CMS'' \\
Brookhaven National Laboratory \\
Brookhaven, NY, US \\
\textbf{Tennessee Tech University Research Seminar for Undergraduates} \>\> Nov. 2018\\ 
``Searching for new physics (and making measurements) at the CERN Large Hadron Collider'' \\
Tennessee Technological University
Cookeville, TN, US \\
\end{tabbing}\vspace{-20pt}      % suppress blank line after tabbing

%\section{Other Experience}

\section{\textsc{Other Professional Experience}}
\vspace{-0.1in}
\begin{tabbing}
\hspace{2.3in}\= \hspace{2.6in}\= \kill % set up two tab positions
\bf{Ultimate precision at hadron colliders workshop, Paris, France}	\>\> Winter 2019 	\\
  Participant \\
\bf{First Electroweak Symmetry Breaking School, Maratea, Italy}	\>\> Spring 2018 	\\
  Student \\
\bf{Multi-boson Interactions 2017, KIT, Karlsruhe, Germany}	\>\> Summer 2017 	\\
  Conference Participant \\
\bf{Physics at TeV Colliders 2017: Les Houches Workshop}	\>\>{Summer 2017 }	\\
  Contributor to Monte Carlo working group \\
  Convener of Vector Boson Scattering Monte Carlo study \\
\bf{Machine Learning in High Energy Physics School, Lund, Sweden}	\>\>{Summer 2016 }	\\
  Student \\
\bf{CTEQ Phenomenology School, DESY, Hamburg, Germany}	\>\>{Summer 2016 }	\\
  Student \\
\bf{MCNET Monte Carlo School, Spa, Belgium}	\>\>{Summer 2015 }	\\
  Student \\
\bf{CMS Data Analysis School (CMSDAS2015@LPC), Batavia, IL}	\>\>{Winter 2015 }\\
  Student in the Mono-photon analysis group (\emph{awarded best project}) 	\\
\\
\bf{CERN-Fermilab Hadron Collider School, Batavia, IL}		\>\>{Summer 2014 }\\
  Student \\
\end{tabbing}\vspace{-20pt}      % suppress blank line after tabbing
\vspace{-0.1in}
\begin{tabbing}
\hspace{2.3in}\= \hspace{2.5in}\= \kill % set up two tab positions
  \textbf{Research Experience for Undergraduates Student Researcher} \>\>{May 2012--Aug. 2012} \\
National Superconducting Cyclotron Laboratory, Symmetry Energy Project Group \\
Michigan State University, East Lansing, MI \\
\end{tabbing}\vspace{-20pt}      % suppress blank line after tabbing

Assisted in construction and design of the SAMURAI Time Projection Chamber. 
Particularly involved in selecting conductive materials for use in active 
volume of detector. Researched and tested products for performance and lack 
of contaminants. Also focused on optimizing performance of the TPC gating grid circuit.

\begin{tabbing}
\hspace{2.3in}\= \hspace{2.5in}\= \kill % set up two tab positions
\textbf{Undergraduate Research Assistant, Astrophysics Group} \>\> May 2011--Aug. 2011\\
Holifield Radioactive Ion Beam Facility \\
Oak Ridge, TN \\
\end{tabbing}\vspace{-20pt}      % suppress blank line after tabbing

Preparation of deuterated polyethylene targets for inverse kinematics 
transfer-reaction measurements. Research into improvement of 
target making process with emphasis on creation of thinner targets. 
Also assisted in preparation of experiment and covering of shifts 
during experiment.

\section{\textsc{Awards}}
\vspace{-0.1in}
\begin{tabbing}
\hspace{2.3in}\= \hspace{2.6in}\= \kill % set up two tab positions
\textbf{Undergraduate Physics Honor Society} \>\>{Spring 2012 }\\
Sigma Pi Sigma Tennessee Technological University Chapter 	    \\
\textbf{Tennessee Technological University Physics Award} 				\>\>{Spring 2013 }\\
Awarded to top graduating physics student \\
\end{tabbing}\vspace{-20pt}      % suppress blank line after tabbing

\section{\textsc{Teaching}}
\vspace{-0.1in}

\begin{tabbing}
\hspace{3.5 in}\= \hspace{1.4in}\= \kill % set up two tab positions
\textbf{Teaching Assistant}, University of Wisconsin-Madison, WI, US \>\> Fall 2013 \\
General Physics 202: Electricity and Magnetism for Engineers \\
TA duties included leading weekly discussions and labs for 50 students. \\
\textbf{Teaching Assistant}, Tennessee Technological University, TN, US \>\> 2012--2013 \\
Calculus Based Physics Lab 1 and 2 \\
Instructed and graded physics lab class for 3 semesters (approx. 20 students each) \\
\end{tabbing}\vspace{-20pt}      % suppress blank line after tabbing

\section{\textsc{Skills}}
\begin{itemize}
\item \textbf{Languages:} English (Native), French (Advanced Proficient)
\item \textbf{Computer Languages:} Experienced in C++, Python, Bash; comfortable in Java, R; \\ familiar with Scala, Perl, Fortran
\item \textbf{Technical Skills:} Scientific knowledge, statistical analysis and machine learning, software design and maintenance, distributed computing, big data, database use, collaboration and strong interpersonal communication skills
\end{itemize} 

\end{resume}

\end{document}
