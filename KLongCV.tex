\documentclass[9pt,a4paper]{moderncv}

% moderncv themes
\moderncvtheme[blue]{classic}                  % optional argument are 'blue' (default), 'orange', 'green', 'red', 'purple', 'grey' and 'roman' (for roman fonts, instead of sans serif fonts)
%\moderncvtheme[green]{classic}                % idem

% character encoding
\usepackage[utf8]{inputenc}                   % replace by the encoding you are using

% adjust the page margins
\usepackage[scale=0.89]{geometry}
\setlength{\hintscolumnwidth}{2.5cm}						% if you want to change the width of the column with the dates
\AtBeginDocument{\setlength{\makecvheadnamewidth}{14 cm}}  % only for the classic theme, if you want to change the width of your name placeholder (to leave more space for your address details
\AtBeginDocument{\recomputelengths}                     % required when changes are made to page layout lengths


% personal data
\firstname{Kenneth}
\familyname{Long}
\title{Physicist and data scientist}               % optional, remove the line if not wanted


\address {\textbf{Nationality}:  American}{\textbf{Birth date}: May 8, 1991}
\extrainfo{Route de Champvigny 51, 1242 Satigny (Switzerland)}
\mobile  {  +41 76 503 9513}         % optional, remove the line if not wanted
%\phone{            (+39)062291044      }                      % optional, remove the line if not wanted                  % optional, remove the line if not wanted
\email{kdlong07@gmail.com}  
\social[linkedin]{kdlong07}
\social[github]{kdlong}

\photo[64pt][0.4pt]{photo2}                         % '64pt' is the height the picture must be resized to, 0.4pt is the thickness of the frame around it (put it to 0pt for no frame) and 'picture' is the name of the picture file; optional, remove the line if not wanted


% to show numerical labels in the bibliography; only useful if you make citations in your resume
\makeatletter
\renewcommand*{\bibliographyitemlabel}{\@biblabel{\arabic{enumiv}}}
\makeatother

% bibliography with mutiple entries
%\usepackage{multibib}
%\newcites{book,misc}{{Books},{Others}}

%\nopagenumbers{}                             % uncomment to suppress automatic page numbering for CVs longer than one page
%----------------------------------------------------------------------------------
%            content
%----------------------------------------------------------------------------------

\begin{document}
\maketitle

%\section{Personal Details}
%\cvitem{First name:}{Claudia}
%\cvitem{Last name:}{Tambasco}
%\cvitem{Nationality:}{Italian}
%\cvitem{Date of birth:}{28th May 1988}
%\cvitem{Country of birth:}{Italy}
%\cvitem{Place of birth:}{Rome (Italy)}
%\cvitem{Home address:}{Via F.Vitalini, 65  E/D}
%\cvitem{}{00155 Roma}

\section{Skills}

\cvline{Programming Languages}{Highly experienced in Python, Bash, C++, C;
    comfortable in R, Java, Scala; familiar with SQL, Perl, Fortran}
\cvline{Analysis}{Linear/logistic regression, mixed linear models, neural networks and deep learning, tree-based methods}
\cvline{Tools}{Machine learning: Keras, Tensorflow; visualization: matplotlib, plotly, ggplot; general: git, HTCondor}
\cvline{Communication}{Experienced communicating work to diverse audiences. Presented 10--20 talks at international conferences. Seminars and volunteer efforts communicating science to general public.}

\section{Experience}
\cventry{Aug 2019 - present}{Senior research fellow}{European Organization for Nuclear Research (CERN)}{Geneva, CH}{}
{
\begin{itemize}
    \item Developing new approaches to object clustering using graph neutral networks. Optimizing algorithms for fast and accurate clustering of hundreds of thousands of electrical signals into clusters of a common origin. Publication in progress.
    \item Developed visualization framework exploiting modern python packages (numpy, pandas, plotly) for intuitive evaluation of clustering algorithms.
    \item Building simulations that enable rigorous data set labeling for unambiguous training and validation of clustering algorithms.
    \item Developed production framework for a lightweight and versatile columnar data format used by hundreds of scientists for faster simulation production and more intuitive analysis. Recognized with a collaboration-wide award for this work.
    \item Using state-of-the art tools implementing TensorFlow to perform complex maximum-likelihood fits with hundreds of bins and thousands of nuisance parameters to precisely estimate data set properties.
    \item Coordinating a group of $\sim$50 colleagues making measurements of predicted, but previously unmeasured, phenomena. Lead the review of 15-20 projects, resulting in $\sim$10 publications per year.
\end{itemize}
}

\cventry{Jul 2013 - Jul 2019}{Researcher}{University of Wisconsin-Madison}{Madison, WI, USA and Geneva, CH}{}
{
\begin{itemize}
    \item Analyzed data set of billions of particle collisions to perform the first measurement of a rare process. Used statistical modeling and maximum likelihood estimation to establish the properties of this new process and their statistical significance.
    \item Developed, validated, and deployed scalable software frameworks to perform big data analysis using
distributed computing infrastructure (100 GB--1 PB pipelines) that led to several publication. Framework currently used by 10-20 people.
    \item Developed and implemented an object-oriented structure for simulation meta data. Integrated into software used by hundreds of colleagues. 
    \item Lead a small group ($\sim$10 people) on the support, software integration, and validation of new simulation tools, exploiting expertise in software, numerical calculations, and distributed computing.
    \item Supported the operation of CERN particle detectors as an on-call expert several weeks per year. Monitored data quality and promptly addressed computing and hardware issues preventing high-quality data collection.
    \item Instructor for two classes of electricity and magnetism for engineers ($\sim$50 students). Responsible for preparing 4 lectures/week and 2 labs/week as well as preparing and grading assignments.
\end{itemize}
}
\newpage
\section{Education}
%\cventry{year--year}{Degree}{Institution}{City}{\textit{Grade}}{Description}
\cventry{July 2013 -- Apr 2019}{Ph.D. in Physics}{University of Wisconsin-Madison}{Madison, WI, USA}{}
{
    Specialization: Experimental particle physics\newline 
    Thesis topic:  \href{http://cds.cern.ch/record/2673625?ln=en}{\emph{Measurement of electroweak WZ boson production and search for new physics in proton-proton collisions at $\sqrt{s}=13$\,TeV with the CMS detector at the CERN LHC}} \\
    \normalsize \textit{GPA: 3.73/4.00} 
}
\cventry{Aug 2009 -- May 2013}{Bachelor of Science in Physics}{Tennessee Technological University}{Cookeville, TN, USA}{}{\normalsize \textit{GPA: 3.97/4.00}}

\section{Languages}
\cvlanguage{English}{Native}{}
\cvlanguage{French}{CEFR B2}{Completed B2 certification course winter 2021}

\section{Personal skills}
\cvline{}{Extensive project management experience, working with and leading diverse groups of collaborators. Experienced at summarizing complex topics to expert and non-expert audiences.}
\end{document}
