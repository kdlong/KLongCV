\documentclass[9pt,a4paper]{moderncv}

% moderncv themes
\moderncvtheme[blue]{classic}                  % optional argument are 'blue' (default), 'orange', 'green', 'red', 'purple', 'grey' and 'roman' (for roman fonts, instead of sans serif fonts)
%\moderncvtheme[green]{classic}                % idem

% character encoding
\usepackage[utf8]{inputenc}                   % replace by the encoding you are using

% adjust the page margins
\usepackage[scale=0.89]{geometry}
\setlength{\hintscolumnwidth}{2.5cm}						% if you want to change the width of the column with the dates
\AtBeginDocument{\setlength{\makecvheadnamewidth}{14 cm}}  % only for the classic theme, if you want to change the width of your name placeholder (to leave more space for your address details
\AtBeginDocument{\recomputelengths}                     % required when changes are made to page layout lengths


% personal data
\firstname{Kenneth}
\familyname{Long}
\title{Physicist and data scientist}               % optional, remove the line if not wanted


\address {\textbf{Nationality}:  American}
{\textbf{Birth date}: May 8, 1991}
\extrainfo{Route de Champvigny 51, 1242 Satigny (Switzerland)}
\mobile  {+41 (0)77 503 9513}         % optional, remove the line if not wanted
\email{kdlong07@gmail.com}  
\social[linkedin]{kdlong07}
\social[github]{kdlong}

% comment out for US CV
\photo[64pt][0.4pt]{photo}                         % '64pt' is the height the picture must be resized to, 0.4pt is the thickness of the frame around it (put it to 0pt for no frame) and 'picture' is the name of the picture file; optional, remove the line if not wanted


% to show numerical labels in the bibliography; only useful if you make citations in your resume
\makeatletter
\renewcommand*{\bibliographyitemlabel}{\@biblabel{\arabic{enumiv}}}
\makeatother

% bibliography with mutiple entries
%\usepackage{multibib}
%\newcites{book,misc}{{Books},{Others}}

%\nopagenumbers{}                             % uncomment to suppress automatic page numbering for CVs longer than one page
%----------------------------------------------------------------------------------
%            content
%----------------------------------------------------------------------------------

\begin{document}
\maketitle

%\section{Personal Details}
%\cvitem{First name:}{Claudia}
%\cvitem{Last name:}{Tambasco}
%\cvitem{Nationality:}{Italian}
%\cvitem{Date of birth:}{28th May 1988}
%\cvitem{Country of birth:}{Italy}
%\cvitem{Place of birth:}{Rome (Italy)}
%\cvitem{Home address:}{Via F.Vitalini, 65  E/D}
%\cvitem{}{00155 Roma}

\vspace{-1cm}
\section{Skills}

\cvline{Programming}{Highly experienced in Python, Bash, C++, C;
    comfortable in SQL, R, Java, Scala}
\cvline{Analysis}{Linear/logistic regression, mixed linear models, deep learning, tree-based methods}
\cvline{Tools}{Machine learning (Keras, Tensorflow, PyTorch), visualisation (matplotlib, plotly, ggplot), data wrangling (pandas, numpy), version control (git), continuous integration (GitHub CI), batch computing (HTCondor, Slurm), data bases (SQL), containerization (Docker)}
\cvline{Communication}{Experienced communicating work to diverse audiences. Presented $\sim$20 talks at international conferences. Seminars and volunteer efforts communicating science to general public.}

\section{Experience}
\cventry{Aug 2019 - present}{Physicist and data scientist}{Massachusetts Institute of Technology (MIT) and European Organization for Nuclear Research (CERN)}{Geneva, CH}{}
{
\begin{itemize}
    \item Leader of a 20 person team maintaining and developing point cloud clustering algorithms.
    \item Developed and deployed custom graph neural network for fast clustering and semantic segmentation of point clouds with $\mathcal{O}$(100k) elements.
        Improved accuracy by $\sim$25\% over algorithmic approach (publication 1). Built plotly-based dashboard to visualize performance.
    \item Principle developer, maintainer, and reviewer of C++ and python-based software framework with $\sim$20 contributors. Utilizing Eigen and TensorFlow for fast data processing and statistical inference.
    \item Implemented and supported self-hosted GitHub CI for automated validation of full processing pipeline. Maintaining custom Docker container for portabilty of software stack.
    \item Coordinated group of $\sim$50 contributors dedicated to rare physics measurements. Lead review of 15-20 projects, resulting in $\sim$20 publications over two years.
\end{itemize}
}

\cventry{Jul 2013 - Jul 2019}{Doctoral researcher}{University of Wisconsin-Madison}{Madison, WI, USA and Geneva, CH}{}
{
\begin{itemize}
    \item Developed, validated, and deployed scalable software frameworks to perform big data analysis using
distributed computing infrastructure (100 GB--1 PB pipelines), used by 10-20 people.
    \item Lead a group ($\sim$10 people) on the support, software integration, and validation of new simulation tools, exploiting expertise in software, numerical calculations, and distributed computing.
    \item Instructor for two introductory physics classes ($\sim$50 students). Prepared 4 lectures/week and 2 labs/week, wrote and graded assignments.
\end{itemize}
}
\section{Education}
%\cventry{year--year}{Degree}{Institution}{City}{\textit{Grade}}{Description}
\cventry{July 2013 -- Apr 2019}{Ph.D. in Physics}{University of Wisconsin-Madison}{Madison, WI, USA}{}
{
    Specialization: Experimental particle physics.
    \normalsize \textit{GPA: 3.73/4.00} 
}
\cventry{Aug 2009 -- May 2013}{Bachelor of Science in Physics}{Tennessee Technological University}{Cookeville, TN, USA}{}{\normalsize \textit{GPA: 3.97/4.00}}

\section{Publications}
\href{https://inspirehep.net/authors/1280606}{Hundreds of peer-reviewed publications} in major journals with the CMS Collaboration ($\sim$10 as principle editor). Several limited-author publications on use of graph neural networks for object clustering, including:
\begin{itemize}
    \item J. Kieseler, K. Long, M. Pierini et al., ``End-to-end multi-particle reconstruction in high occupancy imaging calorimeters with graph neural networks,'' Eur. Phys. J. C 82, 753 (2022), \href{https://arxiv.org/abs/2106.01832}{\texttt{arXiv:2106.01832 [ins-det]}}.
    \item K. Long et al., ``GNN-based end-to-end reconstruction in the CMS Phase 2 High-Granularity Calorimeter,'' J. Phys.: Conf. Ser. 2438 012090, \href{https://arxiv.org/abs/2203.01189 }{\texttt{arXiv:2203.01189 [ins-det]}}.
\end{itemize}

\pagebreak
\section{Awards}
\cvline{2020}{CMS Collaboration Achievement Award: for exceptional contributions to the collaboration}
\cvline{2019}{CERN Research Fellowship: two year research funding at CERN with full academic freedom}
\cvline{2013}{Physics Award: top graduating physics major at Tennessee Technological University}
\section{Languages}
\cvlanguage{English}{Native}{}
\cvlanguage{French}{Professionally proficient}{Completed CEFR B2 certification course winter 2021}

\section{Personal skills}
\cvline{}{Extensive project management experience, working with and leading diverse groups of collaborators. Experienced at summarizing complex topics to expert and non-expert audiences.}
\end{document}
