\documentclass[10pt]{res} % default is 10 pt
%\usepackage{helvetica} % uses helvetica postscript font (download helvetica.sty)
%\usepackage{newcent}   % uses new century schoolbook postscript font 
%\setlength{\textheight}{9 in} % increase text height to fit resume on 1 page
%\newsectionwidth{0pt}  % So the text is not indented under section headings
\usepackage[dvipsnames]{xcolor}
\usepackage{hyperref}
\usepackage[T1]{fontenc}
\usepackage{enumitem}

\setlist[itemize]{align=parleft,left=0pt..1em, label={$-$}, itemindent=0mm,labelsep=1mm}
\setlist[enumerate]{align=parleft,left=0pt..1em, itemindent=0mm,labelsep=1mm}

\newcommand{\mw}{\ensuremath{m_{\mathrm{W}}}}

\begin{document}

\name{Kenneth Long\\ \hspace{-0.2cm}{kenneth.long@cern.ch} \\[11pt]} % the \\[12pt] adds a blank line after name

\address{\textbf{Current Address} \\  CERN B40/2-A20 \\  Geneva 23 \\ Switzerland \\ Tel: +41 77 501 95 13 }
\address{}

\begin{resume}
\section{\textsc{Education}}
\begin{itemize}
    \item \textbf{University of Wisconsin-Madison}, Madison, WI \\
Ph.D. in Experimental High Energy Physics, May 2019 \\
  Thesis advisor: Prof. Matthew Herndon \\
    Thesis topic: \emph{Measurement of electroweak WZ boson production and search for new physics in proton-proton collisions at $\sqrt{s}=13$\,TeV with the CMS detector at the CERN LHC}
    \item \textbf{Tennessee Technological University}, Cookeville, TN \\
Bachelor of Science in Physics, \textit{Summa Cum Laude}, May 2013
\end{itemize}

\section{\textsc{Professional Experience}}
\begin{itemize}
  \item \textbf{Senior Postdoctoral Research Associate} \hfill{Nov. 2021 -- Present} \\
Massachusetts Institute of Technology (MIT)   \\  Boston, MA, US and Meyrin, Switzerland
  \item \textbf{Senior Research Fellow} \hfill{Aug. 2019 -- Oct. 2021} \\
European Organization for Nuclear Research (CERN)   \\  Meyrin, Switzerland
  \item \textbf{Doctoral Researcher} \hfill{Jul. 2013 -- Jul. 2019} \\
Univ. of Wisconsin-Madison Compact Muon Solenoid Group \\  Madison, WI, US and Meyrin, Switzerland
\end{itemize}

\section{\textsc{Principle research objectives and accomplishments}}
\begin{itemize}
  \item\textbf{{Probing the standard model with precise measurements of vector boson production}} \\
    Resolving possible differences in the measured and predicted W boson mass (\mw), which could point to new physics, 
    is key priority of collider physics.
    I have led the development of the analysis strategy, software framework, and precise simulation for the CMS $\mw$ measurement, 
    which will be published in summer 2024 with the best precision from an LHC experiment. My work has demonstrated the 
    power of the CMS data to constrain the Parton Distribution Functions (PDFs) [6], a leading uncertainty in $\mw$ measurements at the LHC,
    and reducing their contribution to the uncertainty in $\mw$ by 40\% compared to other LHC measurements. 
    In close collaboration with theorists, I integrated state-of-the-art calculations directly into the data analysis,
    reducing the impact of the W boson transverse momentum modeling uncertainty in $\mw$ by a factor of 2.
    I also played a crucial role in measuring the W and Z boson cross sections [1] and differential distributions in low pileup. 
    These results validate theoretical predictions used in the $\mw$ measurement and improve future PDFs and perturbative calculations.

  \item\textbf{Elucidating the process of electroweak symmetry breaking via studies of diboson production and vector boson scattering} \\
    I convened the CMS diboson analysis group from 2019-2021, leading the group to $>$10 new publications. 
    Exploiting the full Run 2 data, we observed key standard model processes for the first time, including the first-ever search 
    for longitudinal vector boson scattering (VBS). Without a Higgs boson, 
    the production rate of this process is divergent. Confirming that the Higgs boson fully regulates its rate is a long-term goal at the LHC.
    I made critical contributions to the analysis and simulation for the first measurements of WZ and ZZ production at 13 TeV [10, 11]. 
    I was responsible for all aspects of the first CMS study of WZ vector boson scattering, including the development of the analysis 
    procedure and software tools, performance optimization, and validation [8]. Following my Ph.D., I was closely involved in the 
    development of the extended study for first WZ vector boson scattering analysis in CMS, including the development of a novel 
    approach to simultaneously measure the WZ and same-sign WW electroweak cross sections [7]. I extended these works into detailed 
    projections of the timeline and prospects for the observation of longitudinal VBS at the upgraded high-luminosity LHC [17].

  \item\textbf{Developing novel and precise global event reconstruction algorithms} \\
    I convened the particle flow reconstruction group from fall 2021-2023, overseeing the preparation for and collection of the Run 3 data.
    I maintained and improved global event reconstruction algorithms while coordinating a small team implementing physics 
    and computational performance improvements, work that impacts every analysis performed in the CMS collaboration. 
    I am a lead developer and avid proponent of 
    the use of novel artificial intelligence (AI) techniques for energy clustering. Exploiting the correspondence
    between energy clustering and object identification in computer vision tasks, we demonstrated improved accuracy and computational 
    performance for AI-based clustering over rule-based approaches [13]. I led the application
    of these techniques to reconstruction with the CMS high-granularity calorimeter upgrade,
    as well as the integration into the CMS software framework [3, 4], a critical step to enabling
    precise measurements at the high-luminosity LHC.

\end{itemize}

\section{\textsc{Leadership roles in the CMS Collaboration}}
\begin{itemize}
\item{\textbf{Convener of Particle Flow Reconstruction Group} \hfill{Sep. 2021 -- Aug. 2023}} \\
    Coordinated group of $\sim$30 members maintaining and improving CMS global event reconstruction algorithms.
    Oversaw and implemented critical performance improvements and calibrations for 
    changing detector conditions in preparation for Run 3 data collection.
    Improved hadronic energy calibration workflow and procedure, significantly
    reducing required maintenance and improving low-energy performance by 30\%.
\item{\textbf{Convener of Multi-boson Analysis Group} \hfill{Sep. 2019 -- Aug. 2021}} \\
    Coordinating standard model physics subgroup studying production of final states
    with multiple vector bosons. Oversaw the development and publication of over ten new measurements, 
    including the first observation of triple vector boson production and the most accurate measurements 
    to date of WZ, W$^{\pm}$W$^{\pm}$, and W$\gamma$ vector boson scattering. 
\item{\textbf{Convener of Matrix Element and Future Generators Group} \hfill{Jun. 2017 -- May 2019}}\\
    Directed Monte Carlo Generators subgroup 
    responsible for maintenance and development of simulation software used to simulate
    billions of collisions for use in CMS analyses.
\item{\textbf{Standard Model Physics Group Monte Carlo Contact} \hfill{Feb. 2016 -- Aug. 2017}} \\
    Oversaw production of Monte Carlo event samples for the CMS standard model physics group. 
    Duties included technical support, coordination, and monitoring computing performance.
\item{\textbf{ Data Certification Expert, Cathode Strip Chamber}\hfill{Jul. 2015 -- Nov. 2018}} \\
    Coordinated discussion and determined if
    detector data quality was satisfactory for use in physics analyses.
    Communicated with detector experts to understand
    causes of data quality degradation.
\end{itemize}

\section{\textsc{Other research projects and responsibilities}}
\begin{itemize}
  \item\textbf{Data acquisition, storage manager developer and on-call}\hfill{Fall 2021 -- present} \\
    Part of a three-person team developing software and providing 24/7 support for the merging
    of data files from independent compute nodes of the software trigger system,
    as well as the transfer of files from the CMS detector site to the central storage system
    at the CERN main campus. Utilizing expertise in high-performance shared file systems and multi-threaded
    programming.

  \item\textbf{Monte Carlo generator integration and support}\hfill{Winter 2017 -- Summer 2021} \\
    Developed and implemented the production of a new data format, now used throughout CMS, to enable fast and lightweight production
    of generator-level collision simulations. Performed major restructuring of CMS simulation software interface
    to systematic variations, improving ease of use and enabling more extensive evaluation of theoretical uncertainties in analysis.
    Leading developer of software used to interface simulation tools
    developed by theoretical collaborations into experimental software framework (2017 -- 2019).

  \item\textbf{Endcap muon detector detector on-call shifter}\hfill{Apr. 2016 -- Jul. 2018} \\
    First contact for CMS endcap muon system during commissioning and data-taking for three weeks per year
    from 2016-2018. Monitoring of the cathode strip chamber system, coordinating maintenance activities,
    and addressing urgent issues affecting data collection or quality.

  \item\textbf{Muon detector longevity studies}\hfill{Jan. 2016 -- Jun. 2016} \\
    Assisted in setup of test system and performance measurements of cathode strip chambers
    exposed to high-intensity gamma radiation to assess the physical degradation and impact on muon
    reconstruction performance. Operation shifts during data collection with muon beams.
\end{itemize}

\vspace{-2mm}
\section{\textsc{CMS publications and public results with significant contributions}}
\begin{enumerate}
  \item CMS Collaboration, ``Measurement of W and Z boson inclusive cross sections in pp collisions at 5.02 and 13 TeV,'' \href{https://cds.cern.ch/record/2868090}{\texttt{CMS-PAS-SMP-20-004}} 
  \item CMS Collaboration, ``Measurement of the differential ZZ$+$jets production cross sections in pp collisions at $\sqrt{s} = 13$ TeV,'' \href{https://cds.cern.ch/record/2859350}{\texttt{CMS-PAS-SMP-22-001}} 
  \item S. Bhattacharya, K. Long, et. al. for the CMS Collaboration, ``GNN-based end-to-end reconstruction in the CMS Phase 2 High-Granularity Calorimeter,'' J. Phys.: Conf. Ser. 2438 012090, \href{https://arxiv.org/abs/2203.01189}{\texttt{arXiv:2203.01189 [ins-det]}} 
  \item S. Qasim, K. Long, et. al. for the CMS Collaboration, ``Multi-particle reconstruction in the High Granularity Calorimeter using object condensation and graph neural networks,'' Proceedings for the 25th International Conference on Computing in High-Energy and Nuclear Physics, \href{https://arxiv.org/abs/2106.01832}{\texttt{arXiv:2106.01832 [ins-det]}} 
  \item CMS Collaboration, ``Measurements of pp $\rightarrow$ ZZ production cross section and constraints on anomalous triple gauge couplings at $\sqrt{s} = 13~\mathrm{TeV}$,'' Eur. Phys. J. C 81 (2021) 200, \href{https://arxiv.org/abs/2009.01186}{\texttt{arXiv:2009.01186 [hep-ex]}}
  \item CMS Collaboration, ``Measurements of the W boson rapidity, helicity, double-differential cross sections, and charge asymmetry in pp collisions at $\sqrt{s}=13$\,TeV,'' Phys. Rev. D 102, 092012 (2020), \href{https://arxiv.org/abs/2008.04174}{\texttt{arXiv:2008.04174 [hep-ex]}}
  \item CMS Collaboration, ``Measurements of production cross sections of WZ and same-sign WW boson pairs in association with two jets in proton-proton collisions at $\sqrt{s} =$ 13 TeV'' Phys. Lett. B 809 (2020) 135710, \href{https://arxiv.org/abs/2005.01173}{\texttt{arXiv:2005.01173 [hep-ex]}}
  \item CMS Collaboration, ``Measurement of electroweak WZ boson production and search for new physics in WZ $+$ two jets events in pp collisions at $\sqrt{s}=13$\,TeV,'' Phys. Lett. B 795 (2019) 281, \href{https://arxiv.org/abs/1901.04060} {\texttt{arXiv:1901.04060 [hep-ex]}}
  \item CMS Collaboration, ``Measurement of the pp $\rightarrow$ ZZ production cross section, $\mathrm{Z} \to 4\ell$ branching fraction, and constraints on anomalous triple gauge couplings at $\sqrt{s} = 13~\mathrm{TeV}$,'' Eur. Phys. J. C 78 (2018) 165, \href{https://arxiv.org/abs/1709.08601}{\texttt{arXiv:1709.08601 [hep-ex]}}
  \item CMS Collaboration, ``Measurement of the WZ production cross section in pp collisions at $\sqrt{s}$ = 13 TeV,''
Phys. Lett. B 766, 268 (2016), \href{https://arxiv.org/abs/1607.06943}{\texttt{arXiv:1607.06943 [hep-ex]}}
  \item CMS Collaboration, ``Measurement of the ZZ production cross section and Z $\rightarrow l^{+}l^{-}l'^{+}l'^{-}$ branching fraction in pp collisions at $\sqrt{s} =$ 13 TeV,''
Phys. Lett. B 763, (2016) 280, \\ \href{https://arxiv.org/abs/1607.08834} {\texttt{arXiv:1607.08834 [hep-ex]}}
  \item CMS Collaboration, ``Performance of the CMS muon detector and muon reconstruction with proton-proton collisions at $\sqrt{s} = 13$ TeV,''
JINST 13 (2018) P06015, \href{https://arxiv.org/abs/1804.04528} {\texttt{arXiv:1804.04528 [hep-ex]}}
\end{enumerate}

\section{\textsc{Limited authorship publications with significant contributions}}
\begin{enumerate}\addtocounter{enumi}{12}
  \item S. Qasim, N. Chernyavskaya, J. Kieseler, K. Long, et. al., ``End-to-end multi-particle reconstruction in high occupancy imaging calorimeters with graph neural networks,'' 
    Eur. Phys. J. C 82 (2022) 8, 753, \href{https://arxiv.org/abs/2106.01832}{\texttt{arXiv:2106.01832 [ins-det]}}
  \item D. Franzosi et al.,``Vector Boson Scattering Processes: Status and Prospects,''
    Rev. Phys. 8 (2022) 100071, VBSCAN-PUB-04-21, \href{https://arxiv.org/abs/2106.01393}{\texttt{arXiv:2106.01393 [hep-ph]}}
  \item M. Gallinaro, K. Long, J. Reuter, R. Ruiz (editors) et al.,``Beyond the Standard Model in Vector Boson Scattering Signatures,''
    CERN-OPEN-2020-008, \href{https://arxiv.org/abs/2005.09889}{\texttt{arXiv:2005.09889 [hep-ph]}}
  \item S. Amoroso et al., ``Les Houches 2019: Physics at TeV Colliders Standard Model Working Group Report,''
    \href{https://arxiv.org/abs/2003.01700 } {\texttt{arXiv:2003.01700 [hep-ph]}}
  \item HL-LHC Collaboration and HE-LHC Working Group, ``Standard Model Physics at the HL-LHC and HE-LHC,''
    CERN-LPCC-2018-03, \href{https://arxiv.org/abs/1902.04070}{\texttt{arXiv:1902.04070 [hep-ph]}}
  \item J.R. Andersen et al., ``Les Houches 2017: Physics at TeV Colliders Standard Model Working Group Report,''
    FERMILAB-CONF-18-122-CD-T, UWTHPH-2018-5, \href{https://arxiv.org/abs/1803.07977}{\texttt{arXiv:1803.07977 [hep-ph]}}
\end{enumerate}

\section{\textsc{Presentations at International Conferences and Workshops}}
\begin{enumerate}
  \item ``Electroweak (single W/Z) measurements at ATLAS and CMS,''
    \emph{58th Rencontres de Moriond on Electroweak Interactions and Unified Theories}, Contributed Plenary. Mar. 2024, La Thuile, Italy.
  \item ``Role of theoretical uncertainties in the measurement of $m_{\mathrm{W}}$,''
    \emph{CMS W boson mass measurement Hackathon: Open session}, Invited Plenary. Jan. 2023, Boston, MA, USA.
  \item ``Status and outlook for particle flow reconstruction at CMS,'' Invited Plenary.
    \emph{ECFA topical meeting on reconstruction at Higgs factories}, May 2022, Hamburg, Germany.
  \item ``CMS particle flow status and prospects for Run 3,'' 
    \emph{Ramping up to Run 3: CMS JetMET workshop}, Invited Plenary. Apr. 2022, Florence, Italy.
  \item ``EW precision measurements in single boson production at CMS,'' 
    \emph{32nd Rencontres de Blois 2021}, Contributed Parallel. Oct. 2021, Blois, France.
  \item ``Overview of CMS vector boson scattering results,'' 
    \emph{Vector boson scattering for SNOWMASS---VBSCan workshop}, Invited Plenary. Feb. 2021, Virtual.
  \item ``Vector boson scattering results and future prospects,'' 
    \emph{BSM models in vector boson scattering---VBSCan workshop}, Invited Plenary. Dec. 2019, Lisbon, Portugal.
  \item ``Recent vector boson fusion and scattering measurements at ATLAS and CMS,'' 
    \emph{Standard Model at the LHC 2019}, Contributed Plenary. Apr. 2019, Zurich, Switzerland.
  \item ``Search for electroweak WZ vector boson scattering and new physics at CMS,''
    \emph{US LHC Users Association Meeting 2018}, Young Scientists Lightning Round. Oct. 2018, Fermilab, Batavia, IL, US. \emph{Best presentation award}.
  \item ``Results from ATLAS and CMS on the neutral VBS production,'' 
    \emph{Multi-boson Interactions 2018}, Contributed Plenary. Aug. 2018, Ann Arbor, MI, US.
  \item ``Vector Boson Scattering Results from CMS'' \\
    \emph{XXXIX International Conference on High Energy Physics}, Contributed Parallel, Jul. 2018. Seoul, South Korea.
  \item ``VBS and VBF results from ATLAS and CMS,''
    \emph{30th Rencontres de Blois}, Contributed Parallel. Jun. 2018, Blois, France.
  \item ``Use of Monte Carlo generators in CMS,'' 
    \emph{17th MCnet Collaboration Meeting}, Contributed Plenary, Apr. 2018, CERN, Geneva, Switzerland.
  \item ``Multiboson Results from CMS''
    \emph{Large Hadron Collider Physics 2017}, Contributed Parallel, May 2017. Shanghai, China.
  \item ``Resent results from ATLAS and CMS on the VV+jets production,''
    \emph{Multi-boson Interactions 2016}, Contributed Plenary, Aug. 2016. Madison, WI, US.
  \item ``Materials testing and performance optimization for the SAMURAI-TPC,''
    \emph{Division of Nuclear Physics, American Physical Society Fall Meeting}, Poster presentation. Fall 2012, Newport Beach, CA, US.
  \item ``Creation of Thin Deuterated Polyethylene Targets for Inverse Kinematics Transfer Reaction Measurements,''
    \emph{Division of Nuclear Physics, American Physical Society Fall Meeting}, Poster presentation. Fall 2011. East Lansing, MI, USA.
\end{enumerate}

\section{\textsc{Invited Seminars}}
\begin{enumerate}
 \item ``Measurements and new physics searches with WZ vector boson scattering events at CMS,''
     \emph{Brookhaven National Laboratory Particle Physics Seminar}, Jan. 2019, Brookhaven National Laboratory, Brookhaven, NY, US.
  \item ``Searching for new physics (and making measurements) at the CERN Large Hadron Collider,''
     \emph{Tennessee Tech University Research Seminar for Undergraduates}, Nov. 2018, Tennessee Technological University, Cookeville, TN, US.
\end{enumerate}

\section{\textsc{Other Professional Experience}}
\begin{itemize}
\item \textbf{Research Experience for Undergraduates Student Researcher}\hfill{May 2012 -- Aug. 2012} \\
National Superconducting Cyclotron Laboratory, Michigan State University, East Lansing, MI \\
Contributed to construction and design of the SAMURAI Time Projection Chamber,
especially conductive materials for use in active 
volume of detector. Researched and tested products for performance and lack 
of contaminants. Optimized performance of the TPC gating grid circuit.

\item \textbf{Undergraduate Research Assistant, Astrophysics Group} \hfill{May 2011--Aug. 2011} \\
Holifield Radioactive Ion Beam Facility, Oak Ridge, TN. \\
Preparation of deuterated polyethylene targets for inverse kinematics 
transfer-reaction measurements. Research into improvement of 
target making process with emphasis on creation of thinner targets. 
Assisted in preparation of experiment and covering of shifts 
during experiment.
\end{itemize}

\section{\textsc{Awards}}
\begin{itemize}
  \item \textbf{CMS Collaboration 2020 Achievement Award}\hfill{2020}\\
    For outstanding contributions to the Monte Carlo simulation group
  \item \textbf{Physics Award} 				\hfill{Spring 2013 }\\
    Awarded to top graduating physics student at Tennessee Technological University
\end{itemize}

\section{\textsc{Teaching}}

\begin{itemize}
\item \textbf{Teaching Assistant}, University of Wisconsin-Madison, WI, US \hfill{Fall 2013} \\
Leading weekly discussions and labs for 50 students in electricity and magnetism for engineers.
\item \textbf{Teaching Assistant}, Tennessee Technological University, TN, US \hfill{2012--2013} \\
Instructed and graded introductory physics lab for engineers for 3 semesters (20 students each).
\end{itemize}

\section{\textsc{Skills}}
\begin{itemize}
\item \textbf{Languages:} English (Native), French (CEFR B2 level)
\item \textbf{Computer Languages:} Experienced in C++, Python, Bash; comfortable in Java, R, Scala; \\ familiar with Perl, Fortran
\item \textbf{Technical Skills:} Scientific knowledge, statistical analysis and machine learning, software design and maintenance, distributed computing, big data, database use, collaboration and strong interpersonal communication skills
\end{itemize} 

\end{resume}

\end{document}
