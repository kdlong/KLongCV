\documentclass[10pt]{res} % default is 10 pt
%\usepackage{helvetica} % uses helvetica postscript font (download helvetica.sty)
%\usepackage{newcent}   % uses new century schoolbook postscript font 
%\setlength{\textheight}{9 in} % increase text height to fit resume on 1 page
%\newsectionwidth{0pt}  % So the text is not indented under section headings
\usepackage{hyperref}

\begin{document}

\name{Kenneth Long\\ kdlong@wisc.edu \\[11pt]} % the \\[12pt] adds a blank line after name

\address{{\bf Current Address} \\  CERN 32/4-B03  \\  CH-1211 Geneva 23 \\ Switzerland }
\address{{\bf Permanent Address} \\ 141 County Hill Rd. \\ Blountville,
TN 37617 \\ (423) 323-8326}

\begin{resume}

\section{EDUCATION}
  \textbf{University of Wisconsin-Madison}, Madison, WI \\
Ph. D. in Experimental High Energy Physics, Expected Jan. 2019 \\
Thesis Advisor: Prof. Matthew Herndon \\
\\
  \textbf{Tennessee Technological University}, Cookeville, TN \\
Bachelor of Science in Physics, \textit{Summa Cum Laude}, May 2009 \\

\section{EXPERIENCE}
\vspace{-0.1in}
\begin{tabbing}
\hspace{2.3in}\= \hspace{2.5in}\= \kill % set up two tab positions
{\bf Graduate Student Research Assistant} \>\> 2013-Present \\
Experimental High Energy Physics  (Particle Physics)\\
Univ. of Wisconsin -- Madison Compact Muon Solenoid Group at CERN   \\  Madison, WI and Meyrin, Switzerland \\
\end{tabbing}\vspace{-20pt}      % suppress blank line after tabbing

Focusing on precision standard model measurements and new physics searches in 
diboson channels. Made significant contributions to first 13 TeV measurements of 
WZ and ZZ cross sections at CMS and currently leading analysis efforts 
towards the observation of WZ vector boson scattering. 
Played a leading role in CMS for production and validation of
Monte Carlo simulation for standard model processes.
Also involved in data quality monitoring for the Cathode Strip 
Chambers (CSC) of the CMS muon detector system, CSC detector-on-call shifts, and longevity studies for the CSC at the CERN 
Gamma Irradiation Facility.

\underline{Leadership Roles}
\vspace{2mm}
\begin{itemize}
  \item{Convener of Matrix Element and Future Generators Group (Level 3) \hfill{Sep. 2017 - present}}
  \item{Standard Model Physics Group Monte Carlo Contact \hfill{Feb. 2016 - Aug. 2017}}
  \item{Cathod Strip Chambers Data Certification Expert \hfill{July 2015 - present}}
\end{itemize}

%\underline{Projects and Responsibilities}
%\vspace{2mm}
%\begin{itemize}
%  \item{bonjour}
%\end{itemize}
%
\begin{tabbing}
\hspace{2.3in}\= \hspace{2.5in}\= \kill % set up two tab positions
  \textbf{Research Experience for Undergraduates Student Researcher} \>\>{May 2012 - Aug. 2012} \\
National Superconducting Cyclotron Laboratory, Symmetry Energy Project Group \\
Michigan State University, East Lansing, MI \\
\end{tabbing}\vspace{-20pt}      % suppress blank line after tabbing

Assisted in construction and design of the SAMURAI Time Projection Chamber. 
Particularly involved in selecting conductive materials for use in active 
volume of detector. Researched and tested products for performance and lack 
of contaminants. Also focused on optimizing performance of the TPC gating grid circuit.

\begin{tabbing}
\hspace{2.3in}\= \hspace{2.5in}\= \kill % set up two tab positions
\textbf{Undergraduate Research Assistant, Astrophysics Group} \>\> May 2011 - Aug. 2011\\
Holifield Radioactive Ion Beam Facility \\
Oak Ridge, TN \\
\end{tabbing}\vspace{-20pt}      % suppress blank line after tabbing

Preparation of deuterated polyethylene targets for inverse kinematics 
transfer-reaction measurements. Research into improvement of 
target making process with emphasis on creation of thinner targets. 
Also assisted in preparation of experiment and covering of shifts 
during experiment.

\section{CMS Publications and Public Results with Significant Contributions}
\begin{itemize}
  \item CMS Collaboration, ``Measurement of the pp $\rightarrow$ ZZ production cross section, $\mathrm{Z} \to 4\ell$ branching fraction, and constraints on anomalous triple gauge couplings at $\sqrt{s} = 13~\mathrm{TeV}$'', CMS-PAS-SMP-16-017 (Submitted to Eur. Phys. J. C.), \href{https://arxiv.org/abs/1709.08601}{\texttt{arXiv:1709.08601 [hep-ex]}}
  \item CMS Collaboration, ``Measurement of vector boson scattering and constraints on anomalous quartic couplings from events with four leptons and two jets in proton-proton collisions at $\sqrt{s} =$ 13 TeV''
    Phys. Lett. B 774, (2017) 682, \href{https://arxiv.org/abs/1708.02812}{\texttt{arXiv:1708.02812 [hep-ex]}}
  \item CMS Collaboration, ``Measurement of the WZ production cross section in pp collisions at $\sqrt{s}$ = 13 TeV''
Phys. Lett. B 766, 268 (2016), \href{https://arxiv.org/abs/1607.06943}{\texttt{arXiv:1607.06943 [hep-ex]}}
  \item CMS Collaboration, ``'Measurement of the ZZ production cross section and Z $\rightarrow l^{+}l^{-}l'^{+}l'^{-}$− branching fraction in pp collisions at $\sqrt{s} =$ 13 TeV''
Phys. Lett. B 763, (2016) 280, \href{https://arxiv.org/abs/1607.08834} {\texttt{arXiv:1607.08834 [hep-ex]}}
\end{itemize}

\section{Other Publications}
\begin{itemize}
  %\item J.R. Andersen et al., %``Les Houches 2017: Physics at TeV Colliders Standard Model Working Group Report,''
  %  FERMILAB-CONF-18-122-CD-T, UWTHPH-2018-5
  %\item C. Anders et al., ``VBSCan Split 2017 Workshop Summary,'' 
  %  VBSCAN-PUB-01-17, FERMILAB-CONF-18-021-PPD
  \item Long, Kenneth, ``Multi-boson Measurements in CMS,''
    \href{https://arxiv.org/abs/1710.05673}{\texttt{arXiv:1710.05673 [hep-ex]}}
\end{itemize}

\section{Presentations/Talks}
\vspace{-0.1in}

\begin{tabbing}
\hspace{2.3in}\= \hspace{2.6in}\= \kill % set up two tab positions
%\bf{17th MCnet Collaboration Meeting} 		 \> \>	    Apr. 2017 \\
%CERN, Geneva, Switzerland \\
%Invited Plenary\\ 
%``Use of Monte Carlo Generators in CMS'' \\
\bf{Large Hadron Collider Physics 2017} 		 \> \>	    May 2017 \\
Shanghai, China \\
Contributed Parallel \\ 
``Multiboson Results from CMS'' \\

\bf{Multi-boson Interactions 2016} 		 \> \>	    Aug. 2016 \\
Madison, WI, US \\
Contributed Plenary \\ 
``Resent Results from ATLAS and CMS on the Production of Diboson States \\Associated with Jets'' \\
\bf{Division of Nuclear Physics, American Physical Society Fall Meeting} 		 \> \>	    Fall 2012 \\
Newport Beach, CA \\
Poster Presentation, Conference Experience for Undergraduates Session\\ 
``Materials Testing and Performance Optimization for the SAMURAI-TPC'' \\
\bf{Division of Nuclear Physics, American Physical Society Fall Meeting} 		 \> \>	    Fall 2011 \\
Poster Presentation, Conference Experience for Undergraduates Session\\ 
``Creation of Thin Deuterated Polyethylene Targets for Inverse Kinematics \\Transfer Reaction Measurements'' \\
East Lansing, MI \\
\end{tabbing}

\section{Other Experience}
\vspace{-0.1in}
\begin{tabbing}
\hspace{2.3in}\= \hspace{2.6in}\= \kill % set up two tab positions
\bf{First Electroweak Symmetry Breaking School, Maratea, Italy}		\\
  Student \>\> Spring 2018 \\
\bf{Multi-boson Interactions 2017, KIT, Karlsruhe, Germany}		\\
  Conference Participant \>\> Summer 2017 \\
\bf{Physics at TeV Colliders 2017: Les Houches Workshop}		\\
  Contributor to Monte Carlo working group \>\>{Summer 2017 }\\
  Convener of Vector Boson Scattering Monte Carlo study \\
\bf{Machine Learning in High Energy Physics School, Lund, Sweden}		\\
  Student \>\>{Summer 2016 }\\
\bf{CTEQ Phenomenology School, DESY, Hamburg, Germany}		\\
  Student \>\>{Summer 2016 }\\
\bf{MCNET Monte Carlo School, Spa, Belgium}		\\
  Student \>\>{Summer 2015 }\\
\bf{CMS Data Analysis School (CMSDAS2015@LPC), Batavia, IL}		\\
  Student in the Mono-photon analysis group (awarded best project) \>\>{Jan. 2015 }\\
\bf{CERN-Fermilab Hadron Collider School, Batavia, IL}		\\
  Student \>\>{Summer 2014 }\\
\end{tabbing}
	

\section{Awards}
\vspace{-0.1in}
\begin{tabbing}
\hspace{2.3in}\= \hspace{2.6in}\= \kill % set up two tab positions
\textbf{Undergraduate Physics Honor Society} \>\>{Spring 2012 }\\
Sigma Pi Sigma Tennessee Technological University Chapter 	    \\
\textbf{Tennessee Technological University Physics Award} 				\>\>{Spring 2013 }\\
Awarded to top graduating physics student \\
\end{tabbing}

\section{Teaching}
\vspace{-0.1in}

\begin{tabbing}
\hspace{3.5 in}\= \hspace{1.4in}\= \kill % set up two tab positions
\textbf{Teaching Assistant}, University of Wisconsin -- Madison \>\> Fall 2013 \\
General Physics 202: Electricity and Magnetism for Engineers \\
TA duties included leading weekly discussions and labs for 50 students. \\
\textbf{Teaching Assistant}, Tennessee Technological University \>\> Fall 2012 - Spring 2013 \\
Calculus Based Physics Lab 1 and 2 \\
Instructed and graded physics lab class for 3 semesters (approx. 20 students each). \\
\end{tabbing}

\section{Skills}
\begin{itemize}
\item {\bf Languages:} English (Native),   French (Advanced Proficient)
\item {\bf Computer Languages: } C++,  Python, ROOT, Linux/Unix Shell Scripting 
\item {\bf Technical Skills} Scientific knowledge, software design and maintenance, distributed computing, big data, database use, collaboration and strong interpersonal communication skills, experienced with statistical analysis. 
\end{itemize} 

\end{resume}

\end{document}
