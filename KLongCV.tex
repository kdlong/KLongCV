\documentclass[10pt]{res} % default is 10 pt
%\usepackage{helvetica} % uses helvetica postscript font (download helvetica.sty)
%\usepackage{newcent}   % uses new century schoolbook postscript font 
%\setlength{\textheight}{9 in} % increase text height to fit resume on 1 page
%\newsectionwidth{0pt}  % So the text is not indented under section headings
\usepackage{hyperref}

\begin{document}

\name{Kenneth Long\\ kdlong@wisc.edu \\[11pt]} % the \\[12pt] adds a blank line after name

\address{{\bf Current Address} \\  CERN 32/4-B03  \\  CH-1211 Geneva 23 \\ Switzerland }
\address{{\bf Permanent Address} \\ 141 County Hill Rd. \\ Blountville,
TN 37617 \\ (423) 323-8326}

\begin{resume}

\section{EDUCATION}
  \textbf{University of Wisconsin-Madison}, Madison, WI \\
Ph. D. in Experimental High Energy Physics, Expected Dec. 2018 \\
Thesis Advisor: Prof. Matthew Herndon \\
\\
  \textbf{Tennessee Technological University}, Cookeville, TN \\
Bachelor of Science in Physics, \textit{Summa Cum Laude}, May 2013 \\

\section{EXPERIENCE}
\vspace{-0.1in}
\begin{tabbing}
\hspace{2.3in}\= \hspace{2.5in}\= \kill % set up two tab positions
{\bf Graduate Student Research Assistant} \>\> 2013-Present \\
Experimental High Energy Physics  (Particle Physics)\\
Univ. of Wisconsin -- Madison Compact Muon Solenoid Group at CERN   \\  Madison, WI and Meyrin, Switzerland \\
\end{tabbing}\vspace{-20pt}      % suppress blank line after tabbing

Focusing on precision standard model measurements and new physics searches in 
diboson channels. Made significant contributions to first 13 TeV measurements of 
WZ and ZZ cross sections at CMS and lead analysis efforts 
for first study of WZ vector boson scattering at CMS. 
Played a leading role in CMS for production and validation of
Monte Carlo simulation for standard model processes.
Also involved in data quality monitoring for the Cathode Strip 
Chambers (CSC) of the CMS muon detector system, CSC detector-on-call shifts, and longevity studies for the CSC at the CERN 
Gamma Irradiation Facility (GIF++).

\underline{Leadership Roles}
\vspace{2mm}
\begin{itemize}
  \item{Convener of Matrix Element and Future Generators Group (Level 3) \hfill{Sep. 2017 - present}} \\
    Oversee subgroup of the CMS Generators Physics Object Group 
    responsible for maintence and developements of external matrix element software, 
    including MadGraph5\_aMC@NLO, POWHEG, Sherpa, and Herwig. Responsible for development and support of sample generation for
    major 2017 Monte Carlo production.
  \item{Standard Model Physics Group Monte Carlo Contact \hfill{Feb. 2016 - Aug. 2017}} \\
    Responsible for coordinating and assisting 
    in the production of Monte Carlo samples for the CMS Standard Model Physics Group. Dutings include supporting technical issues
    in generation and computing and coordinating sample requests and submission.
  \item{Cathod Strip Chambers Data Certification Expert \hfill{July 2015 - present}} \\
    Responsible for coordinating and communicating
    whether quality of data from CSC is sufficient for CMS physics analysis.
    Responsibilities include assessing data quality and coordinating with CSC online and offline experts to understand
    causes of CSC data quality degredation.
\end{itemize}

\underline{Analysis Acitivies}
\vspace{2mm}
\begin{itemize}
  \item{First CMS search for WZ production via vector boson scattering}\\
    Analysis contact for first WZ vector boson scattering (VBS) analysis in CMS. 
    Focusing on measurement of standard model production, with limits on dimension 8 
    effective field theory operators and charged Higgs bosons also presented.
    Responsible for all aspects of analysis, 
    including developement of analysis framework, procedure, optimization, and validation. 

  \item{First 13 TeV measurements of WZ and ZZ cross sections} \\
    Contributed to publications in 2016 and 2017 on the first total cross section measurement 
    of ZZ and WZ production at 13 TeV and differential measurements of ZZ production with full 2016 dataset. 
    Responsible for producing acceptance
    and efficiency calculations and validation of all theoretical inputs for both analyses. Developed independent framework
    for cross-checks and validation for WZ analysis.

\end{itemize}

\underline{Other Projects and Responsibilities}
\vspace{2mm}
\begin{itemize}
  \item CSC Detector On-Call Shifts \hfill{Apr. 2016 - Jul. 2018} \\
    First contact person for the CSC system during CMS commissioning and data-taking for three weeks per year
    from 2016-2018. Duties include monitoring of the CSC system, coordinating the activities of CSC with all of CMS, and responding to 
    and addressing any urgent issues affecting CMS data collection or quality.

  \item CMS MadGraph5\_aMC@NLO integration and support \hfill{Jan. 2017 - present} \\
    Developer of CMS central scripts
    for configuring and steering MadGraph5\_aMC@NLO, the most-used Monte Carlo generator in CMS. Responsible for 
    providing user support, implementing new 
    developements into the CMS framework, and for improving usability and computing robustness. Principle developer during production
    of new Monte Carlo samples for 2017 data taking.

  \item Multi-boson contact person for CMS Cross Section Task Force \hfill{Oct. 2015 - Mar. 2017} \\
    Served as liason with theory community to understand best-available calculations and 
    their application to CMS Monte Carlo samples. 

  \item CSC longevity studies at GIF++. \hfill{Jan. 2016 - Jun. 2016} \\
    Assisted in setup and performance measurements
    of cathode strip chambers exposed to high intensity gamma radiation to assess the physical degredation and impact on performance
    of muon reconstruction and triggering at CMS. Primarily responsible for measurements over time of leakage current
    versus instantaneous and accumulated radiation. Contributed to shifts during data collection with muon beams.
\end{itemize}

\begin{tabbing}
\hspace{2.3in}\= \hspace{2.5in}\= \kill % set up two tab positions
  \textbf{Research Experience for Undergraduates Student Researcher} \>\>{May 2012 - Aug. 2012} \\
National Superconducting Cyclotron Laboratory, Symmetry Energy Project Group \\
Michigan State University, East Lansing, MI \\
\end{tabbing}\vspace{-20pt}      % suppress blank line after tabbing

Assisted in construction and design of the SAMURAI Time Projection Chamber. 
Particularly involved in selecting conductive materials for use in active 
volume of detector. Researched and tested products for performance and lack 
of contaminants. Also focused on optimizing performance of the TPC gating grid circuit.

\begin{tabbing}
\hspace{2.3in}\= \hspace{2.5in}\= \kill % set up two tab positions
\textbf{Undergraduate Research Assistant, Astrophysics Group} \>\> May 2011 - Aug. 2011\\
Holifield Radioactive Ion Beam Facility \\
Oak Ridge, TN \\
\end{tabbing}\vspace{-20pt}      % suppress blank line after tabbing

Preparation of deuterated polyethylene targets for inverse kinematics 
transfer-reaction measurements. Research into improvement of 
target making process with emphasis on creation of thinner targets. 
Also assisted in preparation of experiment and covering of shifts 
during experiment.

\section{CMS Publications and Public Results with Significant Contributions}
\begin{itemize}
  \item CMS Collaboration, ``Measurement of electroweak WZ production and search for new physics in pp collisions at $\sqrt{s} =$ 13 TeV'', CMS-PAS-SMP-18-001
  \item CMS Collaboration, ``Measurement of the pp $\rightarrow$ ZZ production cross section, $\mathrm{Z} \to 4\ell$ branching fraction, and constraints on anomalous triple gauge couplings at $\sqrt{s} = 13~\mathrm{TeV}$'', Eur. Phys. J. C 78 (2018) 165, \href{https://arxiv.org/abs/1709.08601}{\texttt{arXiv:1709.08601 [hep-ex]}}
  \item CMS Collaboration, ``Measurement of the WZ production cross section in pp collisions at $\sqrt{s}$ = 13 TeV''
Phys. Lett. B 766, 268 (2016), \href{https://arxiv.org/abs/1607.06943}{\texttt{arXiv:1607.06943 [hep-ex]}}
  \item CMS Collaboration, ``'Measurement of the ZZ production cross section and Z $\rightarrow l^{+}l^{-}l'^{+}l'^{-}$ branching fraction in pp collisions at $\sqrt{s} =$ 13 TeV''
Phys. Lett. B 763, (2016) 280, \href{https://arxiv.org/abs/1607.08834} {\texttt{arXiv:1607.08834 [hep-ex]}}
\end{itemize}

\section{Other Publications}
\begin{itemize}
  \item J.R. Andersen et al., ``Les Houches 2017: Physics at TeV Colliders Standard Model Working Group Report,''
    FERMILAB-CONF-18-122-CD-T, UWTHPH-2018-5
  \item C. Anders et al., ``VBSCan Split 2017 Workshop Summary,'' 
    VBSCAN-PUB-01-17, FERMILAB-CONF-18-021-PPD
  \item Long, Kenneth, ``Multi-boson Measurements in CMS,''
    Proceedings for Large Hadron Collider Physics 2017. CMS-CR-2017-364
\end{itemize}

\section{Presentations/Talks}
\vspace{-0.1in}

\begin{tabbing}
\hspace{2.3in}\= \hspace{2.6in}\= \kill % set up two tab positions
\bf{XXXIX International Conference on High Energy Physics} 		 \> \>	    Jul. 2018 \\
``Vector Boson Scattering Results from CMS'' \\
Seoul, South Korea\\
\bf{30th Rencontres de Blois} 		 \> \>	    Jun. 2018 \\
``VBS and VBF Results from ATLAS and CMS'' \\
Blois, France\\
Contributed Parallel\\ 
\bf{17th MCnet Collaboration Meeting} 		 \> \>	    Apr. 2018 \\
CERN, Geneva, Switzerland \\
Contributed Plenary\\ 
``Use of Monte Carlo Generators in CMS'' \\
\bf{Large Hadron Collider Physics 2017} 		 \> \>	    May 2017 \\
Shanghai, China \\
Contributed Parallel \\ 
``Multiboson Results from CMS'' \\

\bf{Multi-boson Interactions 2016} 		 \> \>	    Aug. 2016 \\
Madison, WI, US \\
Contributed Plenary \\ 
``Resent Results from ATLAS and CMS on the Production of Diboson States \\Associated with Jets'' \\
\bf{Division of Nuclear Physics, American Physical Society Fall Meeting} 		 \> \>	    Fall 2012 \\
Newport Beach, CA \\
Poster Presentation, Conference Experience for Undergraduates Session\\ 
``Materials Testing and Performance Optimization for the SAMURAI-TPC'' \\
\bf{Division of Nuclear Physics, American Physical Society Fall Meeting} 		 \> \>	    Fall 2011 \\
Poster Presentation, Conference Experience for Undergraduates Session\\ 
``Creation of Thin Deuterated Polyethylene Targets for Inverse Kinematics \\Transfer Reaction Measurements'' \\
East Lansing, MI \\
\end{tabbing}

\section{Other Experience}
\vspace{-0.1in}
\begin{tabbing}
\hspace{2.3in}\= \hspace{2.6in}\= \kill % set up two tab positions
\bf{First Electroweak Symmetry Breaking School, Maratea, Italy}		\\
  Student \>\> Spring 2018 \\
\bf{Multi-boson Interactions 2017, KIT, Karlsruhe, Germany}		\\
  Conference Participant \>\> Summer 2017 \\
\bf{Physics at TeV Colliders 2017: Les Houches Workshop}		\\
  Contributor to Monte Carlo working group \>\>{Summer 2017 }\\
  Convener of Vector Boson Scattering Monte Carlo study \\
\bf{Machine Learning in High Energy Physics School, Lund, Sweden}		\\
  Student \>\>{Summer 2016 }\\
\bf{CTEQ Phenomenology School, DESY, Hamburg, Germany}		\\
  Student \>\>{Summer 2016 }\\
\bf{MCNET Monte Carlo School, Spa, Belgium}		\\
  Student \>\>{Summer 2015 }\\
\bf{CMS Data Analysis School (CMSDAS2015@LPC), Batavia, IL}		\\
  Student in the Mono-photon analysis group (awarded best project) \>\>{Jan. 2015 }\\
\bf{CERN-Fermilab Hadron Collider School, Batavia, IL}		\\
  Student \>\>{Summer 2014 }\\
\end{tabbing}
	

\section{Awards}
\vspace{-0.1in}
\begin{tabbing}
\hspace{2.3in}\= \hspace{2.6in}\= \kill % set up two tab positions
\textbf{Undergraduate Physics Honor Society} \>\>{Spring 2012 }\\
Sigma Pi Sigma Tennessee Technological University Chapter 	    \\
\textbf{Tennessee Technological University Physics Award} 				\>\>{Spring 2013 }\\
Awarded to top graduating physics student \\
\end{tabbing}

\section{Teaching}
\vspace{-0.1in}

\begin{tabbing}
\hspace{3.5 in}\= \hspace{1.4in}\= \kill % set up two tab positions
\textbf{Teaching Assistant}, University of Wisconsin -- Madison \>\> Fall 2013 \\
General Physics 202: Electricity and Magnetism for Engineers \\
TA duties included leading weekly discussions and labs for 50 students. \\
\textbf{Teaching Assistant}, Tennessee Technological University \>\> Fall 2012 - Spring 2013 \\
Calculus Based Physics Lab 1 and 2 \\
Instructed and graded physics lab class for 3 semesters (approx. 20 students each). \\
\end{tabbing}

\section{Skills}
\begin{itemize}
\item {\bf Languages:} English (Native),   French (Advanced Proficient)
\item {\bf Computer Languages: } C++,  Python, Java, ROOT, Linux/Unix Shell Scripting 
\item {\bf Technical Skills} Scientific knowledge, software design and maintenance, distributed computing, big data, database use, collaboration and strong interpersonal communication skills, experienced with statistical analysis. 
\end{itemize} 

\end{resume}

\end{document}
