\documentclass[a4paper]{article}
\usepackage{orcidlink}
\usepackage[margin=1.1in]{geometry}
\usepackage[T1]{fontenc}
\usepackage{titling}

\setlength{\droptitle}{-8em}   % This is your set screw
\begin{document}

\title{List of selected publications}
\author{Kenneth Long \orcidlink{0000-0003-0664-1653}}

\maketitle

\noindent Author of more than 800 publications with the CMS Collaboration, a selection of which is given below. Full list available in \href{https://inspirehep.net/authors/1280606}{Inspire} or ORCID\orcidlink{0000-0003-0664-1653}. \\

\section{\textsc{Selected CMS Publications and public results with significant contributions}}
\begin{enumerate}
  \item CMS Collaboration. ``Measurement of W and Z boson inclusive cross sections in pp collisions at 5.02 and 13 TeV,'' 
    \href{https://cds.cern.ch/record/2868090}{\texttt{CMS-PAS-SMP-20-004}}, Submission to JHEP expected in 2024.

    Development of analysis strategy, statistical analysis, and software tools. Integration of next-to-next-to-leading order (NNLO) simulation.
    Extension of analysis to differential distributions for validation of theoretical predictions.

  \item CMS Collaboration, ``Measurement of the differential ZZ+jets production cross sections in pp collisions at $\sqrt{s} = 13$\,TeV,'' 
    \href{https://cds.cern.ch/record/2859350}{\texttt{CMS-PAS-SMP-22-001}}, Submitted to JHEP.

    Extension of analysis framework developed for publication 5 to four-lepton final state and full Run II data set.
    Support and guidance of graduate student in analysis implementation. 
    Produced and validated predictions at NNLO QCD $\times$ NLO electroweak using the MATRIX 
    and MiNNLO$_{PS}$ Monte Carlo programs.

  \item CMS Collaboration, ``Measurements of production cross sections of WZ and same-sign WW boson pairs in association with two jets in proton-proton collisions at $\sqrt{s} =$ 13\,TeV'' Phys. Lett. B 809 (2020) 135710, \href{https://arxiv.org/abs/2005.01173}{\texttt{arXiv:2005.01173 [hep-ex]}}

    Development of novel simultaneous measurement strategy. 
    Implementation of a public Rivet analysis routine, enabling Monte Carlo prediction studies to be easily made 
    with the published results, for the first time in a CMS vector boson scattering measurement
    Guidance of the analysis strategy as convener of the multi-boson analysis group.
  \item CMS Collaboration, ``Measurement of electroweak WZ boson production and search for new physics in WZ $+$ two jets events in pp collisions at $\sqrt{s}=13$\,TeV,'' Phys. Lett. B 795 (2019) 281, \href{https://arxiv.org/abs/1901.04060} {\texttt{arXiv:1901.04060 [hep-ex]}}

    Analysis contact and lead analyzer. Responsible for all aspects, 
    including development of new analysis framework, design of analysis approach including signal extraction and optimization,
    data-driven background estimation, and Monte Carlo modeling validation. 
  \item CMS Collaboration, ``Measurement of the pp $\rightarrow$ ZZ production cross section, $\mathrm{Z} \to 4\ell$ branching fraction, and constraints on anomalous triple gauge couplings at $\sqrt{s} = 13$\,TeV,'' Eur. Phys. J. C 78 (2018) 165, \href{https://arxiv.org/abs/1709.08601}{\texttt{arXiv:1709.08601 [hep-ex]}}

    Produced and validated predictions at NNLO QCD using the MATRIX Monte Carlo generator and at NLO QCD using the MCFM Monte Carlo program.
    Extensive validation of both codes, as acknowledged by the MATRIX authors in subsequent pubications.
    Performed the first differential comparisons of CMS diboson results to NNLO predictions.

  \item CMS Collaboration, ``Measurement of the WZ production cross section in pp collisions at $\sqrt{s}$ = 13\,TeV,''
Phys. Lett. B 766, 268 (2016), \href{https://arxiv.org/abs/1607.06943}{\texttt{arXiv:1607.06943 [hep-ex]}}

    Calculation and validation of theoretical acceptance and experimental efficiency. 
    Analysis software development and nonprompt lepton background estimation.
\end{enumerate}

\section{\textsc{Selected Limited authorship publications with significant contributions}}
\begin{enumerate}\addtocounter{enumi}{6}
  \item S. Qasim, N. Chernyavskaya, J. Kieseler, K. Long, et. al., ``End-to-end multi-particle reconstruction in high occupancy imaging calorimeters with graph neural networks,'' 
    Eur. Phys. J. C 82 (2022) 8, 753, \href{https://arxiv.org/abs/2106.01832}{\texttt{arXiv:2106.01832 [ins-det]}}

    Conceptualization of publication scope and goals, including the use of a simplified detector setup for proof-of-principle performance demonstration.
    Algorithm training and performance evaluation. Supervision of students, writing and editing draft.

  \item S. Bhattacharya, K. Long, et. al. for the CMS Collaboration, ``GNN-based end-to-end reconstruction in the CMS Phase 2 High-Granularity Calorimeter,'' J. Phys.: Conf. Ser. 2438 012090, \href{https://arxiv.org/abs/2203.01189}{\texttt{arXiv:2203.01189 [ins-det]}} 

    Built interactive visualization framework for performance assessment and publication figures.
    Data set production and definition of target clustering truth. 
    Integration and implementation of object condensation technique into CMS software framework. 
     
  \item M. Gallinaro, K. Long, J. Reuter, R. Ruiz (editors) et al.,``Beyond the Standard Model in Vector Boson Scattering Signatures,''
    CERN-OPEN-2020-008, \href{https://arxiv.org/abs/2005.09889}{\texttt{arXiv:2005.09889 [hep-ph]}}

    Convened session and documented discussion in the Lisbon workshop.
    Prepared overview of topics discussed, as well as a summary of existing and prospective measurements and relevant models of new physics in 
    close collaboration with theoreticians. 
    Responsible for experimental component of the document but closely involved in editing all parts.
  \item HL-LHC Collaboration and HE-LHC Working Group, ``Standard Model Physics at the HL-LHC and HE-LHC,''
    CERN Yellow Reports: Monographs, Vol. 7 (2019), CERN-LPCC-2018-03, \href{https://arxiv.org/abs/1902.04070}{\texttt{arXiv:1902.04070 [hep-ph]}}

    Extrapolated the CMS WZ vector boson scattering result (publication 4) to the HL-LHC conditions and luminosity. Supervised junior
    graduate student and extended software tools to enable this study. Investigated variables most sensitive to 
    polarized WZ scattering.
\end{enumerate}

\end{document}
