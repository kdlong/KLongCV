\documentclass[10pt]{res} % default is 10 pt
\usepackage{hyperref}
\begin{document}
\begin{resume}

\section{\textsc{CMS Publications and Public Results with Significant Contributions}}
\begin{enumerate}
  \item S. Qasim, K. Long, J. Kieseler, and M. Pierini for the CMS Collaboration, R. Nawaz, ``Multi-particle reconstruction in the High Granularity Calorimeter using object condensation and graph neural networks,'' \href{https://arxiv.org/abs/2106.01832}{\texttt{arXiv:2106.01832 [ins-det]}} 
     
    Integration and implementation of object condensation technique into CMS software framework. 
    Data set production and definition of target clustering truth. Algorithm training and performance evaluation.
  \item CMS Collaboration, ``Measurements of pp $\rightarrow$ ZZ production cross section and constraints on anomalous triple gauge couplings at $\sqrt{s} = 13~\mathrm{TeV}$,'' Eur. Phys. J. C 81 (2021) 200, \href{https://arxiv.org/abs/2009.01186}{\texttt{arXiv:2009.01186 [hep-ex]}}

    Extension of analysis framework developed for publication 4 to four-lepton final state and full Run II data set.
    Support and guidance of graduate student in analysis implementation. 
    Produced and validated predictions at NNLO QCD $\times$ NLO electroweak using the MATRIX Monte Carlo program.
    Predictions used for acceptance calculation and comparison with unfolded results for the first time in a CMS diboson result. 
  \item CMS Collaboration, ``Measurements of production cross sections of WZ and same-sign WW boson pairs in association with two jets in proton-proton collisions at $\sqrt{s} =$ 13 TeV'' Phys. Lett. B 809 (2020) 135710, \href{https://arxiv.org/abs/2005.01173}{\texttt{arXiv:2005.01173 [hep-ex]}}

    Development of novel simultaneous measurement strategy. 
    Implementation of a public Rivet analysis routine, enabling Monte Carlo prediction studies to be easily made 
    with the published results, for the first time in a CMS vector boson scattering measurement
    Guidance of the analysis strategy as convener of the multi-boson analysis group.
  \item CMS Collaboration, ``Measurement of electroweak WZ boson production and search for new physics in WZ $+$ two jets events in pp collisions at $\sqrt{s}=13$\,TeV,'' Phys. Lett. B 795 (2019) 281, \href{https://arxiv.org/abs/1901.04060} {\texttt{arXiv:1901.04060 [hep-ex]}}

    Analysis contact and lead analyzer. Responsible for all aspects, 
    including development of new fast analysis framework, design of analysis approach including signal extraction and optimization,
    data-driven background estimation, and Monte Carlo modeling validation. 
  \item CMS Collaboration, ``Measurement of the pp $\rightarrow$ ZZ production cross section, $\mathrm{Z} \to 4\ell$ branching fraction, and constraints on anomalous triple gauge couplings at $\sqrt{s} = 13~\mathrm{TeV}$,'' Eur. Phys. J. C 78 (2018) 165, \href{https://arxiv.org/abs/1709.08601}{\texttt{arXiv:1709.08601 [hep-ex]}}

    Produced and validated predictions at NNLO QCD using the MATRIX Monte Carlo generator and at NLO QCD using the MCFM Monte Carlo program.
    Extensive validation of both codes, acknowledged by the authors in subsequent pubications.
    Performed the first differential comparisons of CMS diboson results to NNLO predictions.
  \item CMS Collaboration, ``Measurement of the WZ production cross section in pp collisions at $\sqrt{s}$ = 13 TeV,''
Phys. Lett. B 766, 268 (2016), \href{https://arxiv.org/abs/1607.06943}{\texttt{arXiv:1607.06943 [hep-ex]}}

    Calculation and validation of theoretical acceptance and experimental efficiency. 
    Analysis software development and nonprompt lepton background estimation.
  \item CMS Collaboration, ``Measurement of the ZZ production cross section and Z $\rightarrow l^{+}l^{-}l'^{+}l'^{-}$ branching fraction in pp collisions at $\sqrt{s} =$ 13 TeV,''
Phys. Lett. B 763, (2016) 280, \\ \href{https://arxiv.org/abs/1607.08834} {\texttt{arXiv:1607.08834 [hep-ex]}}
JINST 13 (2018) P06015, \href{https://arxiv.org/abs/1804.04528} {\texttt{arXiv:1804.04528 [hep-ex]}}

    Production and extensive validation of diboson simulation with POWHEG and MadGraph5\_aMC@NLO programs. 
    Calculation and validation of theoretical acceptance and experimental efficiency. 
\end{enumerate}

\section{\textsc{Other Publications with Significant Contributions}}
\begin{enumerate}
  \item D. Franzosi et al.,``Vector Boson Scattering Processes: Status and Prospects,''
    Submitted to Reviews in Physics, VBSCAN-PUB-04-21, \href{https://arxiv.org/abs/2106.01393}{\texttt{arXiv:2106.01393 [hep-ph]}}

    Part II primary author and editor
  \item M. Gallinaro, K. Long, J. Reuter, R. Ruiz (editors) et al.,``Beyond the Standard Model in Vector Boson Scattering Signatures,''
    VBSCAN-PUB-04-20, CERN-OPEN-2020-008, \href{https://arxiv.org/abs/2005.09889}{\texttt{arXiv:2005.09889 [hep-ph]}}

    Lead and documented discussion in the Lisbon workshop.
    Produced overview of topics discussed, as well as a summary of existing and prospective measurements and relevant models of new physics in 
    close collaboration with theoreticians. 
    Responsible for experimental component of the document but closely involved in editing all parts.
  \item HL-LHC Collaboration and HE-LHC Working Group, ``Standard Model Physics at the HL-LHC and HE-LHC,''
    CERN-LPCC-2018-03, \href{https://arxiv.org/abs/1902.04070}{\texttt{arXiv:1902.04070 [hep-ph]}}

    Extrapolated the CMS WZ vector boson scattering result (publication 4) to the HL-LHC conditions and luminosity. Supervised junior
    graduate student and extended software tools to enable this study. Investigated variables most sensitive to 
    polarized WZ scattering.
  \item J.R. Andersen et al., ``Les Houches 2017: Physics at TeV Colliders Standard Model Working Group Report,''
    FERMILAB-CONF-18-122-CD-T, UWTHPH-2018-5, \href{https://arxiv.org/abs/1803.07977}{\texttt{arXiv:1803.07977 [hep-ph]}}

    Proposed and lead extensive comparison of Monte Carlo generator predictions of WZ vector boson scattering.
    Produced results in collaboration with authors of major Monte Carlo generators and identified sources of differences.
    This work was crucial to understanding where predictions for vector boson scattering give reliable results,
    and how they can be reliably exploited in experimental analyses.
\end{enumerate}

\end{resume}
\end{document}
